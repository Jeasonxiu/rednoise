%%
%% Beginning of file 'sample.tex'
%%
%% Modified 2005 December 5
%%
%% This is a sample manuscript marked up using the
%% AASTeX v5.x LaTeX 2e macros.

%% The first piece of markup in an AASTeX v5.x document
%% is the \documentclass command. LaTeX will ignore
%% any data that comes before this command.

%% The command below calls the preprint style
%% which will produce a one-column, single-spaced document.
%% Examples of commands for other substyles follow. Use
%% whichever is most appropriate for your purposes.
%%
%%\documentclass[12pt,preprint]{aastex}

%% manuscript produces a one-column, double-spaced document:


\documentclass{aastex}
\usepackage{graphicx}
\usepackage{color}
\usepackage{multirow}
%\usepackage{showkeys}
\usepackage{epsfig}
\usepackage{natbib}
\usepackage{wasysym}
%\usepackage{subfigure}
\usepackage{verbatim}

\bibliographystyle{apj}	%% AASTeX

%% preprint2 produces a double-column, single-spaced document:

%% \documentclass[preprint]{aastex}

%% Sometimes a paper's abstract is too long to fit on the
%% title page in preprint2 mode. When that is the case,
%% use the longabstract style option.

%% \documentclass[preprint2,longabstract]{aastex}

%% If you want to create your own macros, you can do so
%% using \newcommand. Your macros should appear before
%% the \begin{document} command.
%%
%% If you are submitting to a journal that translates manuscripts
%% into SGML, you need to follow certain guidelines when preparing
%% your macros. See the AASTeX v5.x Author Guide
%% for information.

\newcommand{\vdag}{(v)^\dagger}
\newcommand{\myemail}{jack.ireland@nasa.gov}
\newcommand{\PS}{power spectrum}
\newcommand{\mPS}{mean power spectrum}

\newcommand{\PA}{power spectra}
\newcommand{\mPA}{mean power spectra}

\newcommand{\PL}{power-law}

\newcommand{\Fps}{Fourier \PS}
\newcommand{\mFps}{mean Fourier \PS}

\newcommand{\Fpa}{Fourier \PA}
\newcommand{\mFpa}{mean Fourier \PA}

%% You can insert a short comment on the title page using the command below.

%\slugcomment{Not to appear in Nonlearned J., 45.}

%% If you wish, you may supply running head information, although
%% this information may be modified by the editorial offices.
%% The left head contains a list of authors,
%% usually a maximum of three (otherwise use et al.).  The right
%% head is a modified title of up to roughly 44 characters.
%% Running heads will not print in the manuscript style.

\shorttitle{Coronal Fourier power spectra}
\shortauthors{Ireland et al.}

%% This is the end of the preamble.  Indicate the beginning of the
%% paper itself with \begin{document}.

\begin{document}

%% LaTeX will automatically break titles if they run longer than
%% one line. However, you may use \\ to force a line break if
%% you desire.

\title{Coronal Fourier power spectra: implications for coronal
  seismology and coronal heating.}

%% Use \author, \affil, and the \and command to format
%% author and affiliation information.
%% Note that \email has replaced the old \authoremail command
%% from AASTeX v4.0. You can use \email to mark an email address
%% anywhere in the paper, not just in the front matter.
%% As in the title, use \\ to force line breaks.

\author{J. Ireland}
\affil{ADNET Systems, Inc.,NASA Goddard Space Flight Center, MC. 671.1, Greenbelt, MD 20771, USA.}

\author{R. T. J. McAteer}
\affil{Department of Astronomy, New Mexico State University, Las
  Cruces, NM.}

\author{A. R. Inglis}
\affil{NASA Goddard Space Flight Center, Greenbelt, MC. 671.1, MD, 20771, USA.}
\affil{Physics Department, The Catholic University of America, Washington, DC, 20664, USA.}

%% Mark off your abstract in the ``abstract'' environment. In the manuscript
%% style, abstract will output a Received/Accepted line after the
%% title and affiliation information. No date will appear since the author
%% does not have this information. The dates will be filled in by the
%% editorial office after submission.

\begin{abstract}
The \mFpa\ of regions of the solar corona are investigated using AIA
171\AA\ and 193\AA\ data.  It is shown that the \mFpa\ in all regions
show an approximate \PL\ behaviour.  This average behavior is
inconsistent with the common assumption that coronal time-series are
well described by the sum of a long time-scale background trend plus
Gaussian-distributed noise, with some specific locations also showing
an oscillatory signal.  The implications of this discovery to the
field of coronal seismology and automated detections of oscillations
is discussed.  Further, it is shown that the observed \PL\ \PS\ can be
formed by summing a distribution of exponentially decaying emission
events along the line of sight.  This is consistent with the notion
that the solar atmosphere is heated everywhere by small energy
deposition events.
\end{abstract}

%% Keywords should appear after the \end{abstract} command. The uncommented
%% example has been keyed in ApJ style. See the instructions to authors
%% for the journal to which you are submitting your paper to determine
%% what keyword punctuation is appropriate.

\keywords{???}

%% From the front matter, we move on to the body of the paper.
%% In the first two sections, notice the use of the natbib \citep
%% and \citet commands to identify citations.  The citations are
%% tied to the reference list via symbolic KEYs. The KEY corresponds
%% to the KEY in the \bibitem in the reference list below. We have
%% chosen the first three characters of the first author's name plus
%% the last two numeral of the year of publication as our KEY for
%% each reference.


%% Authors who wish to have the most important objects in their paper
%% linked in the electronic edition to a data center may do so by tagging
%% their objects with \objectname{} or \object{}.  Each macro takes the
%% object name as its required argument. The optional, square-bracket 
%% argument should be used in cases where the data center identification
%% differs from what is to be printed in the paper.  The text appearing 
%% in curly braces is what will appear in print in the published paper. 
%% If the object name is recognized by the data centers, it will be linked
%% in the electronic edition to the object data available at the data centers  
%%
%% Note that for sources with brackets in their names, e.g. [WEG2004] 14h-090,
%% the brackets must be escaped with backslashes when used in the first
%% square-bracket argument, for instance, \object[\[WEG2004\] 14h-090]{90}).
%%  Otherwise, LaTeX will issue an error. 

\section{Introduction}\label{sec:int}
Coronal seismology is the study of oscillatory phenomena in the solar
corona.  First suggested by \cite{1970PASJ...22..341U}, the practice
of coronal seismology attempts to locate oscillatory phenomena in the
solar corona, identify the wave modes present, and then use
theoretical descriptions of those wave modes to infer the physical
conditions of the solar corona.  The practice of coronal seismology
began in earnest with observations of wave modes by SOHO-EIT
\citep{1998ApJ...501L.217D, 1999SoPh..186..207B} and TRACE
\citep{1999ApJ...520..880A} (see also \cite{lrsp-2005-3},
\cite{2012RSPTA.370.3193D} for more extensive reviews of coronal
seismology).  Since then, many papers have been published describing
oscillatory phenomena in the corona and their interpretation in terms
of theoretically known wave modes.

Oscillatory phenomena in the solar corona are relatively rare, both
spatially and temporally.  Two types of oscillatory motions have
received a lot of attention.  The first is known as a transverse
oscillation, since the loop motion is approximately transverse to the
extent of the loop \citep{1999Sci...285..862N}.  These motions appear
to be triggered by nearby flaring events, and are identified with the
fundamental harmonic of the kink mode of the flux tube that comprises
the coronal loop.  The second type is known as a longitudinal
oscillation, or propagating disturbance (PD).  These waves propogate
along flux tubes, and have been identified as slow mode (MHD)
magneto-acoustic waves (although alternative interpretations do exist
that do not invoke oscillatory phenomena, for example,
\cite{0004-637X-722-2-1013} and \cite{2041-8205-727-2-L37}).

Their scarcity on the Sun has prompted efforts to create automated
oscillation detection algorithms.  The need for an automated
oscillation detection algorithm is also exacerbated by the large and
rapidly growing amount of AIA data, which means that the amount of
data that has to be searched is large.  There have been several
attempts at designing such algorithms, all of which operate on
time-series of two-dimensional image data (data cubes).
\cite{2004SoPh..223....1D} describe the design of an automated
oscillation-detection algorithm based on the wavelet analysis. The
algorithm finds significant wave packets ranging from single to
multiple wave cycles in duration, by a wavelet power/confidence level
comparison against the null hypothesis that a given time series is
Gaussian-distributed noise.  Pixels with significant oscillatory
content are grouped manually.  \cite{2007SoPh..241..397N} take a
similar approach, using a thresholded fast Fourier transform to find
locations in TRACE data that may support an oscillatory signal. The
threshold level is defined as three to four times the average FFT
power; if the maximum FFT power is above this level then the frequency
at which that power occurs is assumed to be real.
\cite{2010SoPh..264..403I} assume that an oscillatory signal is
present in each location in the datacube and then use a Bayesian-based
treatment to calculate the probability that the frequency has a
particular value.  This algorithm relies on subtracting a background
trend from the time-series being tested, and assumes that the noise is
Gaussian distributed.

The above algorithms precipitate detections based on the strength of
the signal of an oscillation on a pixel by pixel basis, and then group
these pixels together using some set of criteria.  The automated
detection algorithm of \cite{2008SoPh..248..395S} begins by
identifying candidate oscillatory regions as those that show a high
variance \citep{2003SoPh..213..103G} (under the implicit assumption
that the time-series has an approximately constant mean value).
Detections are based on wavelet filtering of these regions in all
three dimensions of the data cube simultaneously, with the implicit
assumption that the noise is Gaussian-distributed.  The algorithm of
\cite{2008SoPh..252..321M} differs from the previous algorithms: it
begins by Fourier transforming the entire data cube, and calculating
the coherence of neighboring pixels in narrow frequency bands, and
filtering the results. The null hypothesis here is that the time
series is pure Gaussian noise, with the additional assumption that
pairs of noisy time-series have low coherence, and so can be rejected.
In this algorithm, the spatial extent of the coherence of the
oscillatory signal is the key identifier of a wave process.

All these detections algorithms, and many detections in the literature
make either the explicit or implicit assumption that the observed
time-series consists of a background trend of some kind and at least
approximately Gaussian-distributed noise.  However, it is shown below,
using data from the Atmospheric Imaging Assembly (AIA)
\cite{2012SoPh..275...17L} on board the Solar Dynamics Observatory
(SDO) \cite{2012SoPh..275....3P}, that time-series of coronal emission
may be better described by other models.  These models have
implications for the practice of coronal seismology, and may be
indicative of the nature of the fundamental energy deposition that
keeps the corona hot.


\section{Observations}\label{sec:obs}
The data used in this study is from the Atmospheric Imaging Assembly
(AIA) \cite{2012SoPh..275...17L} on board the Solar Dynamics
Observatory (SDO) \cite{2012SoPh..275....3P}.  AIA observes in
multiple wavebands simultaneously and continuously, at high cadence
(12 seconds) and at high spatial resolution (0.6 arcseconds per pixel
at the Sun).  These data enable detailed comparisons between phenomena
seen in multiple wavebands, and are therefore ideal for assessing the
frequency content of the solar atmosphere.  The data was acquired
using the Lockheed Martin Solar Astrophysics cutout service
(http://lmsal.com/get\_aia\_data/).  Six hours of image data in the
time range 2012-09-23 00:00 - 06:00 UT, at the full 12 second cadence,
in the 171\AA and 193\AA\ wavebands is used.  These wavebands were
selected because previous studies in similar wavebands (TRACE 171\AA
and 193\AA) show the presence of coronal oscillations.  The cutout of
the Sun selected covers two adjacent $\beta\gamma$ regions NOAA AR
11575 and AR 11577.  This cutout was selected for two reasons.  First,
it is quiescent in that there were no X-ray flares or other
large-scale disturbances over the spatial extent of the cutout within
the time range.  Therefore, any variations in the plasma properties
cannot be ascribed to large-scale disturbances.  Second, the selected
cutout region contains a sunspot, and sunspots are known to be
locations over which three minute oscillations have been detected by
other instruments \citep{2002A&A...387L..13D}.

The downloaded cutout data was prepared for analysis using the
SolarSoft / IDL routines READ\_SDO and AIA\_PREP.  These procedures
convert the downloaded level 1.0 FITS cutout files to level 1.5 FITS
files.  This involves translating, scaling and rotating the images so
they have the same sun-center, image scale with solar north in the
same direction.  All further processing from this point was performed
in the SunPy 0.4 (http://www.sunpy.org) data analysis environment.
Images were de-rotated to compensate for solar rotation, based on the
center of the field of view of each image.  After de-rotation, each
image layer was co-registered with the image half way through the
dataset, at approximately 03:00 UT.

\section{Analysis}\label{sec:anal}
We wish to analyze the frequency content of these resulting datacubes
as a function of spatial location.  Four locations were chosen in the
data representing four different types of physical locations in the
solar atmosphere.  These were a quiet Sun region, a region that lies
on top of a sunspot, a footpoint region, and a coronal moss region.
Figures \ref{fig:loc171193} show the location of the regions on sample
images from the time range considered.

\begin{figure}
\centerline{
\plottwo{shutdownfun3_6hr_disk_1.5_171.eps}{shutdownfun3_6hr_disk_1.5_193.eps}
}
\caption{Locations of the regions analyzed in each of AIA 171\AA\ and
  193\AA, overlaid on sample images taken halfway through the six hour
  span used to form the time-series analyzed.  The spatial extent of
  the data is (-501'', -138'') in the x-direction, and (-80'', 107'')
  in the y-direction, in heliocentric cartesian co-ordinates.
  \citep{2006A&A...449..791T}.}
\label{fig:loc171193}
\end{figure}

Consider the emission from a single pixel, $I(t)$.  The Fourier power
spectrum for this emission was calculated as follows.  First, the
time-series was normalized by calculating $(I(t) - \langle I(t)
\rangle)/\langle I(t) \rangle$. Secondly, the normalized time-series
was apodized using the Hanning window, reducing ringing effects in the
\PS.  The full 12 second cadence is used, yielding time series of 1800
samples.  The full spatial resolution is also retained in order to
obtain fine detail on the variation of frequency content as a function
of location.  Figures \ref{fig:compare171193} show example power
spectra, plotted using a log-log scale, from a single location inside
each of the regions shown in Figure \ref{fig:loc171193}.  The shape of
the power spectra indicate that the Fourier power approximately
follows a \PL\ at lower frequencies, and flattens out at higher
frequencies.

\begin{figure}
\centerline{
\plottwo{loopfootpoints.centralpower.eps}{qs.centralpower.eps}
}
\centerline{
\plottwo{sunspot.centralpower.eps}{moss.centralpower.eps}
}
\caption{Log-log plot of the Fourier power of a time-series at the AIA
  pixel overlying a randomly chosen pixel in each of the regions
  considered (see Figure \protect\ref{fig:loc171193}).  The vertical
  black lines indicate the 3 and 5 minute frequencies.}
\label{fig:compare171193}
\end{figure}

\begin{figure}
\centerline{
\epsscale{0.8}\plottwo{shutdownfun3_6hr_disk_1.5_171_loopfootpoints.hanning.relative.power_spectra_distributions.eps}{shutdownfun3_6hr_disk_1.5_193_loopfootpoints.hanning.relative.power_spectra_distributions.eps}
}
\centerline{
\epsscale{0.8}\plottwo{shutdownfun3_6hr_disk_1.5_171_moss.hanning.relative.power_spectra_distributions.eps}{shutdownfun3_6hr_disk_1.5_193_moss.hanning.relative.power_spectra_distributions.eps}
}
\centerline{
\epsscale{0.8}\plottwo{shutdownfun3_6hr_disk_1.5_171_qs.hanning.relative.power_spectra_distributions.eps}{shutdownfun3_6hr_disk_1.5_193_qs.hanning.relative.power_spectra_distributions.eps}
}
\centerline{
\epsscale{0.8}\plottwo{shutdownfun3_6hr_disk_1.5_171_sunspot.hanning.relative.power_spectra_distributions.eps}{shutdownfun3_6hr_disk_1.5_193_sunspot.hanning.relative.power_spectra_distributions.eps}
}
\caption{Distribution of Fourier power at some sample frequencies as a
function of AIA waveband and location.}
\label{fig:dist171193}
\end{figure}
Figure \ref{fig:dist171193} show the distributions of the logarithm of
the Fourier power for a small number of frequencies, for all the
pixels in each region.  These distributions are centrally peaked, and
show a slightly heavier tail towards lower Fourier powers (negative
skew), and are slightly broader when compared to a Gaussian
distribution.  Plots of all these distribution show that they are all
unimodal.  These observations guide the choice of how to average the
\PS\ over each of the selected region.  The nature of these
distributions suggest that the arithmetic mean of the Fourier power is
biased by the very largest powers.  Therefore, the mean value of the
logarithm of the Fourier power is a better description of the range of
values in the data (this is equivalent to the geometric mean of the
Fourier power).  The \PA\ used below are calculated from the mean
value of the logarithm of the Fourier power.

\begin{figure}
\centerline{
\epsscale{0.8}\plottwo{shutdownfun3_6hr_disk_1.5_171_loopfootpoints.hanning.relative.logiobs.1.model_fit_compare.pymc.eps}{shutdownfun3_6hr_disk_1.5_193_loopfootpoints.hanning.relative.logiobs.1.model_fit_compare.pymc.eps}
}
\centerline{
\epsscale{0.8}\plottwo{shutdownfun3_6hr_disk_1.5_171_moss.hanning.relative.logiobs.1.model_fit_compare.pymc.eps}{shutdownfun3_6hr_disk_1.5_193_moss.hanning.relative.logiobs.1.model_fit_compare.pymc.eps}
}
\centerline{
\epsscale{0.8}\plottwo{shutdownfun3_6hr_disk_1.5_171_qs.hanning.relative.logiobs.1.model_fit_compare.pymc.eps}{shutdownfun3_6hr_disk_1.5_193_qs.hanning.relative.logiobs.1.model_fit_compare.pymc.eps}
}
\centerline{
\epsscale{0.8}\plottwo{shutdownfun3_6hr_disk_1.5_171_sunspot.hanning.relative.logiobs.1.model_fit_compare.pymc.eps}{shutdownfun3_6hr_disk_1.5_193_sunspot.hanning.relative.logiobs.1.model_fit_compare.pymc.eps}
}
\caption{Geometric mean of the \protect\Fpa\ in each of the
  four regions studied in AIA 171\AA\ (left column) and AIA
  193\AA\ (right column).  Also shown are the posterior mean and
  maximum likelihood fits for model $M_{1}$ and $M_{2}$ (see Section
  \ref{sec:anal} and Appendix \ref{sec:app:ind} for more detail).
  Model fit parameter values and their 68\% credible intervals are
  quoted in Table \ref{tab:parameters}.}
\label{fig:fit171193}
\end{figure}

\begin{deluxetable}{cccccccc}
\tabletypesize{\scriptsize}
\tablecolumns{5}
\tablewidth{0pt}
\tablecaption{Parameter values and uncertainty estimates derived for
    model \protect$M_{2}$.\label{tab:parameters}}
\tablehead{
\colhead{Region} & 
\colhead{Waveband}    & 
\multicolumn{3}{c}{\PL} & \multicolumn{3}{c}{lognormal} \\
\colhead{}          & 
\colhead{}          & 
\colhead{$\log_{10}A$}          & 
\colhead{$n$}      & 
\colhead{$\log_{10}C$}      &
\colhead{$\log_{10}\alpha$} &
\colhead{$\beta$}      &
\colhead{$\sigma$}\\
\colhead{}          & 
\colhead{}          & 
\colhead{}          & 
\colhead{}      & 
\colhead{}      &
\colhead{} &
\colhead{(mHz)}      &
\colhead{(decades of frequency)}\\
}
\startdata
loop footpoints & 171\AA  & $0.65^{+0.02}_{-0.02}$  &$2.27^{+0.02}_{-0.02}$  & $-4.15^{+0.00}_{-0.00}$  & $-2.16^{+0.12}_{-0.11}$ & $0.64^{+0.10}_{-0.10}$  & $0.33^{+0.02}_{-0.02}$  \\
                & 193\AA  & $0.21^{+0.02}_{-0.02}$  & $2.27^{+0.03}_{-0.03}$ & $-4.26^{+0.00}_{-0.00}$  & $-2.24^{+0.09}_{-0.10}$ & $0.37^{+0.06}_{-0.04}$  & $0.40^{+0.01}_{-0.02}$  \\
                & & & & & & & \\

moss            & 171\AA  & $0.47^{+0.01}_{-0.01}$  & $1.86^{+0.01}_{-0.01}$ & $-3.97^{+0.01}_{-0.00}$  & $-1.33^{+0.02}_{-0.02}$  & $1.05^{+0.03}_{-0.02}$  & $0.31^{+0.00}_{-0.01}$  \\
                & 193\AA  & $0.57^{+0.01}_{-0.01}$  & $2.09^{+0.01}_{-0.01}$ & $-4.28^{+0.00}_{-0.00}$  & $-1.97^{+0.02}_{-0.03}$  & $1.43^{+0.03}_{-0.03}$  & $0.25^{+0.00}_{-0.00}$  \\
                & & & & & & & \\

quiet Sun       & 171\AA  & $0.67^{+0.01}_{-0.01}$  & $1.73^{+0.01}_{-0.01}$ & $-2.86^{+0.00}_{-0.00}$  & $-3.49^{+0.04}_{-0.04}$  & $4.50^{+0.14}_{-0.09}$  & $0.10^{+0.01}_{-0.01}$  \\
                & 193\AA  & $0.06^{+0.01}_{-0.01}$  & $2.25^{+0.01}_{-0.01}$ & $-3.38^{+0.00}_{-0.00}$  & $-4.54^{+0.06}_{-0.10}$  & $4.72^{-0.17}_{-0.16}$  & $0.13^{+0.01}_{-0.02}$  \\
                & & & & & & & \\

sunspot         & 171\AA  & $0.51^{+0.01}_{-0.01}$  & $2.27^{+0.02}_{-0.02}$ & $-3.79^{+0.00}_{-0.00}$  & $-2.94^{+0.02}_{-0.02}$  & $2.18^{+0.08}_{-0.07}$  & $0.29^{+0.01}_{-0.01}$  \\
                & 193\AA  & $0.02^{+0.02}_{-0.01}$  & $2.19^{+0.02}_{-0.04}$ & $-3.85^{+0.00}_{-0.00}$  & $-3.17^{+0.04}_{-0.04}$  & $1.87^{+0.11}_{-0.11}$  & $0.35^{+0.01}_{-0.01}$  \\
\enddata
%% Text for table notes should follow after the \enddata but before
%% the \end{deluxetable}. Make sure there is at least one \tablenotemark
%% in the table for each \tablenotetext.
\end{deluxetable}

The (geometric) \mFpa\ are shown in Figure \ref{fig:fit171193} for
each of the physical regions considered.  It is clear that, in
general, the \mFpa\ exhibit power-law like characteristics.  The
results for the quiet Sun, loop footpoints and sunspot regions suggest
a \PL\ \PS.  This can be modeled as
\begin{equation}
\label{eqn:pwrlaw}
\mbox{Model $M_{1}$}: P_{1}(\nu) = A\nu^{-n} + C,
\end{equation}
where $\nu$ is the normalized frequency, $A>0$ is a proportionality
constant and $n>0$ is the \PL\ index.  The quantity $C>0$ models
the high-frequency / low power end of the spectrum where the detection
properties of the detector apparatus are assumed to dominate the
observations.

The moss results for 171\AA\ and 193\AA\ are notably different from
the other results.  In comparison to the general trend observed in
other regions, there appears to be excess Fourier power in the range
1-10 mHz.  \cite{2003ApJ...595L..63D} show that TRACE
171\AA\ (TRansition Region And Coronal Explorer) time-series of bright
upper transition region emission above active region plage (also known
as moss) is a source of wavelet power. Periods of significant wavelet
power are found in the range 200 to 600 seconds, and typically persist
for 4–7 cycles.  Later work by \cite{2005ApJ...624L..61D} suggested
that spicule flux tubes tilted with respect to the solar surface
provide a mechanism by which oscillatory power from lower in the
atmosphere may be channeled to upper portions of the atmosphere.  This
suggests that a second model $M_{2}$ for the \Fps\ in regions should
be considered.  The second model $M_{2}$ has two contributions to the
overall \Fps; a background power-law of the form of $P_{1}$, and a
contribution that is more localized, $G(\nu)$.  The model is given by
\begin{equation}
\label{eqn:pwrlawbump}
\mbox{Model $M_{2}$}: P_{2}(\nu) = P_{1}(\nu) + G(\nu)
\end{equation}
where the localized contribution is described by
\begin{equation}
\label{eqn:bump}
G(\nu) = \alpha\exp\left[-\frac{(\ln(\nu)-\beta)^{2}}{2\sigma^{2}}\right].
\end{equation}
Both these models are fit to the geometrically averaged Fourier power
spectrum in both wavelengths and in all four regions (Figure
\ref{fig:fit171193}), and compared to decide which model best
describes the data. The details of the model fitting process are
described in \ref{sec:app:ind}.


\subsection{Results}\label{ssec:results}
Figure \ref{fig:fit171193} show the fit results of each model to the
data.  Table \ref{tab:parameters} shows the fitted parameter values
and an error estimate for model $M_{2}$.  For details on the fitting
procedure, see Section \ref{sec:app:ind}.

Model selection is implemented using the Akaike Information Criterion
(AIC) and the Bayesian Information Criterion (BIC)
\citep{2007MNRAS.377L..74L}.  The AIC is defined as
\begin{equation}\label{eqn:aic}
AIC \equiv -2 \ln L_{max} - 2k
\end{equation}
where $L_{max}$ is the maximum likelihood achievable by the model, and
$k$ is the number of parameters in the model ($k=3$ for Model 1, and
$k=6$ for Model 2).  The BIC is defined as
\begin{equation}\label{eqn:bic}
BIC \equiv -2 \ln L_{max} - k\ln N
\end{equation}
where $N$ is the number of data points used in the fit (in this case
$N=899$, the number of non-zero frequencies of the Fourier power
spectrum).  Model selection is guided by calculating $\delta AIC =
AIC_{1} - AIC_{2}$ and $\delta BIC = BIC_{1} - BIC_{2}$.  Positive
values of $\delta AIC, \delta BIC$ express a preference for model 2
over model 1.

Using both the AIC and the BIC, Model 2 is overwhelmingly preferred to
Model 1 (see Figure \ref{fig:fit171193}). Table \ref{tab:parameters}
gives the parameter estimates and 68\% credible intervals of the
parameter values.  The power law indices for all regions and both
wavebands lie in the range 1.8 to 2.3, although there is no
consistency between one waveband and another.  For example, the
171\AA\ and 193\AA\ loop footpoint results both have the same power
law index, but the other regions do not.  However, it should be noted
that there is no {\it a priori} reason why the power law indices for a
region seen in two different wavebands should be the same.  This is
because each waveband is designed to examine different features on the
Sun in broadly different temperature ranges, and may be imaging very
different physical processes: hence, the emission seen in one waveband
need not have the same statistical properties as any other.



\section{Discussion}
The presence of a \PL\ \PS\ has implications for the detection of
narrow frequency band oscillations against such a background emission,
for both non-automated and automated methods.  These are discussed in
Sections \ref{ssec:corseis} and \ref{sec:oscdetect}.  Further, the
\PL\ Fourier spectrum in coronal emission poses questions about how
that emission is formed.  A hypothesis regarding the formation of this
power law spectrum is presented in Section \ref{ssec:nplps}.

\subsection{Effect of background \PS\ assumptions on the
  detection of oscillatory power}
\label{ssec:corseis}

Section \ref{sec:anal} demonstrates the presence of an approximate
\PL\ \PS\ signal in two of the most commonly used wavebands for
coronal seismology, SDO-AIA 171\AA\ and 193\AA.  The observed spectra
generally show a \PL-like behavior at time-scales of interest in
coronal seismology, and longer.  Within the confines of the models we
have chosen, coronal moss regions show the largest contribution from a
non \PL\ \PS\ source over a relatively limited range of frequencies.
The frequencies at which the lognormal contribution is larger than the
underlying power-law distribution could be used to define an average
frequency range within which the majority contribution to the overall
Fourier power is not associated with the underlying power-law
distribution.  However, the \Fps\ in this range would still have to
analyzed using both the power-law and lognormal contributions to the
overall power, and better estimates to the power-law parameters will
be obtained if the full frequency range of the observations are used.

The presence of a non \PL\ contribution to the \Fps\ does not change
the fact that the background emission is essentially power-law like.
For the frequencies that have been of interest in coronal seismology,
the background \PS\ is therefore scale free on average.  If there is
no time-scale that can be used to remove a background trend, then the
time-series has to be analyzed without this processing step.  However,
a lack of awareness of the background \PL\ \PS\ nature of the emission
can cause problems in attempting to analyze for the presence of narrow
frequency band oscillations, as the following example demonstrates.

\begin{figure}
\epsscale{.80}
\plotone{white_red_compare.eps}
\caption{Comparison of the effect of assuming a white or
  \PL\ \PS\ background on the detection of an oscillatory signal for
  simulated time-series.  Plot (a) shows the simulated time-series,
  generated from a \protect\PS\ \protect$P(f)\propto f^{-1.77}$ and no
  explicit oscillation included.  Plot (b) shows the wavelet power
  spectrum with the cone-of-influence (shaded area) and regions above
  the 95\% confidence level, assuming a white-noise (Gaussian)
  background .  Plot (c) shows the global wavelet \protect\PS\ for
  this wavelet transform.  Plots (d) and (e) are the same as plots (b)
  and (c) under the assumption of a
  \protect\PL\ \protect\PS\ background.}
\label{fig:comparison}
\end{figure}

Figure \ref{fig:comparison} shows how a time series generated from a
random sample from a \PL\ \PS\ can be mistakenly thought to contain a
narrow-band oscillatory signal.  The time series (Figure
\ref{fig:comparison}(a)) is constructed from a \PS\ $P(f)\approx
f^{-1.77}$ with no explicit oscillatory content, following the
construction procedure of \cite{2010MNRAS.402..307V}.  Figure
\ref{fig:comparison}(b) shows that the assumption of a white-noise
background and a 95\% confidence level leads to the positive detection
of significant oscillatory power in this time series. Figure
\ref{fig:comparison}(c) shows that the assumption of a background
\PL\ \PS\ with the same confidence level significantly reduces the
area for which a detection may be claimed.  Even with this
significantly different background assumption, there is still a
significant amount of wavelet power above the 95\% confidence level.
However, since the data is simulated without an explicit oscillatory
component, the high wavelet power must have arisen by chance.  The
simulated data, along with the results of Section \ref{sec:anal},
suggest that when examining the wavelet transforms of time-series for
wave packets, a background \PL\ \PS\ should be assumed, along with
higher confidence levels (99\% or higher), in order to minimize the
effects of mistakenly identifying random variations in the background
\PS\ as evidence for an oscillatory signal.

However, even this approach to identifying oscillatory power is not
quite complete, as it simply rejects the null hypothesis that the
\PS\ can be adequately explained by a \PL\ \PS.

Model comparison techniques similar to that described in Section
\ref{sec:anal} are required to prefer a model with oscillatory content
over one without. The difficulty in finding narrow-band features in
Fourier power spectra against a background \PL\ \PS\ has been
recognized by \cite{2010MNRAS.402..307V} in the study of XMM-Newton
observations of highly variable Seyfert 1 galaxies.
\cite{2010MNRAS.402..307V} presents a model fitting and comparison
technique that is directly applicable to the search for oscillatory
power in coronal time-series.

\subsection{Implications for coronal seismology and automated
  oscillation detection algorithms}\label{sec:oscdetect}

The discussion of Section \ref{ssec:corseis} shows that \PL\ power
spectra have a powerful effect on the claims of a detection of
narrow-frequency band oscillations in single time-series.  When
looking at extended regions of coronal imaging data, evidence for the
presence of a {\it wave} in the corona is strengthened if it can be
shown that narrow frequency-band oscillations occur significantly
above the background signal in several neighboring pixels.  Spatial
extent is the key feature to claim of a detected wave.  Coronal
seismology derives its observational evidence from such detections,
and automated detection algorithms hold out the hope of greatly
increasing the number of oscillating regions detected.

Figure \ref{fig:compare171193} shows that time series from individual
pixels have power-law like power spectra.  This has implications for
the design of automated oscillation detection algorithms.  Such an
algorithm must take into account the nature of the \PS.  The power-law
\PS\ itself lacks an easily definable time-scale at or below which a
background trend can be defined.  Automated detection algorithms that
rely on background trend subtraction, such as
\cite{2010SoPh..264..403I} must therefore look to other reasons for to
justify a time-scale.

As has been shown above, the assumption of a Gaussian-distributed
(white) noise when a \PL\ \PS\ is actually present, coupled with too
low a confidence level, can lead to misleading results when looking
for oscillatory signals.  Many wave detection algorithms do make this
assumption, for example \cite{2004SoPh..223....1D},
\cite{2007SoPh..241..397N}, \cite{2008SoPh..248..395S},
\cite{2010SoPh..264..403I} and \cite{2013SoPh..286..405C}.  Such
approaches have yet to be tested assuming a background \PL\ \PS, and
so their efficacy is unknown.  However, \cite{2004SoPh..223....1D} and
\cite{2008SoPh..248..395S} are wavelet-based approaches, and the
discussion Section \ref{ssec:corseis} does suggest that a background
\PL\ \PS\ will make a significant difference to the number of wave
detections found in the data.

All the automated detection algorithms described above attempt to find
individual pixels that are obviously oscillating at some frequency,
using the strength of the oscillation at that frequency (compared to
some assumption of the background) as the key quantity to measure.
These pixels are then clumped together and the quality of the signal
is then assessed.  Detection algorithms based on {\it wave coherence},
such as \cite{2008SoPh..252..321M}, implement wave detection through
finding spatially contiguous regions of the corona which are coherent
in a given frequency range.  The frequency range searched is based on
previous experience.  Oscillations that are detected propagating along
loops are often found by considering distance-time plots
\citep{2000AA...355L..23D, 2003AA...404L...1K} and looking for
alternating brightening/darkening in those plots as a indicator of a
propagating wave.  This is essentially a coherence-based approach that
involves an implicit assessment of the likelihood that an oscillation
has been detected.  The basis of the assessment is a judgement that a
signal is seen in multiple contiguous pixels.  However, the results of
Section \ref{sec:anal} suggest that each pixel in these plots has a
background \PL\ \PS: the discussion of Section \ref{ssec:corseis}
therefore suggests that the significance of any claimed oscillation is
much reduced.


\subsection{``Excess'' emission}\label{ssec:excess}
The models fits of Section \ref{sec:anal} indicate that equation
\ref{eqn:pwrlaw} is preferred over equation \ref{eqn:pwrlawbump} as
the superior description of the observed Fourier \PS. The
moss region shows the greatest contribution from the lognormal
distribution $G(\nu)$ over the background \PL.
\cite{2005ApJ...624L..61D} show that it is possible to leak
oscillatory power from lower in the atmosphere to the upper reaches of
the atmosphere through the simple expedient of tilting the field line.
This changes the acoustic cutoff frequency and so allows wave energy
to propagate up in to the upper atmosphere.  If spectral model $M_{2}$
represents conditions in the solar atmosphere, then it should be
possible to infer a distribution of field line angles with the solar
surface using \cite{2005ApJ...624L..61D} and the observed power
distribution $G(\nu)$.  This would constrain the distribution of field
line angles in the lower atmosphere in different regions of the solar
surface.


\subsection{Nature of the \protect\PL\ \protect\PS}\label{ssec:nplps}

The first contribution to both models $P_{0}$ and $P_{i}$ is a
\PL-like \PS.  It is present in both wavebands studied and in all the
regions, and so parsimony suggests that the mechanism of its creation
is both similar and ubiquitous in all parts of the solar atmosphere.
We hypothesize that the \PL\ \PS\ is due to the sum of a distribution
of a large number of emission events having different amounts of
emission. \cite{2011soca.book.....A} describes how a \PL\ \PS\ may
obtained from such a distribution, and that argument is outlined
below.

Each emission event is modeled as an exponentially decaying function
of time $t$
\begin{equation}
\label{eqn:expdecay}
f(t) = \frac{E}{T}\exp\left(-\frac{t}{T}\right),
\end{equation}
for some timescale $T$ and emission $E$.  The corresponding \Fps\ is
\begin{equation}
\label{eqn:ftexpdecay}
P(\nu) = \frac{E}{1 + (2\pi \nu T)^{2}}.
\end{equation}
The total \Fps\ along the line of sight is given by
\begin{equation}
\label{eqn:sumftexpdecay}
P_{total}(\nu) = \sum_{T}N(T)P_{T}(\nu)
\end{equation}
where $N(T)$ is the distribution of time-scales for all the events
along the line of sight.  Further, if the number of events of a given
emission $E$ is assumed to be
\begin{equation}
\label{eqn:energydistrib}
N(E) \propto E^{-\alpha_{E}}
\end{equation}
and the total emission in each event depends on its time scale $T$
such that
\begin{equation}
\label{eqn:energytime}
E \propto T^{1+\gamma}
\end{equation}
then it can be shown that the observed \PS\ can be
approximated by
\begin{equation}
\label{eqn:finalfps}
P_{total}(\nu) \propto \nu^{-(2-\alpha_{E})(1+\gamma)}.
\end{equation}
This derivation shows it is possible to generate \PL\ power spectra
using swarms of statistically similar emission events.  

This derivation of a \PL\ \PS\ shares many similar features with the
nanoflare model of {\it energy} deposition in the solar atmosphere. In
the nanoflare model advanced by \cite{1988ApJ...330..474P}, large
numbers of small magnetic reconnection events convert the energy in
the magnetic field into energy that heats the corona.  Nanoflares are
hypothesized to be small ($10^{24}-10^{27}$ ergs) yet ubiquitous in
the corona.  \cite{2013ApJ...771...21W} and \cite{2013ApJ...770L...1T}
analyzing narrowband 193\AA\ images taken by the Hi-C instrument on
board a sounding rocket demonstrate the heating of small scale coronal
structures that appear to be consistent with scenarios of nanoflare
heating due to reconnecting magnetic loops.  However, the nanoflare
occurence rate, or their energy distribution for the entire solar
atmosphere is as yet unknown.

Any modeling or theoretical effort to infer the occurence rate and
energy distribution of energy deposition events from the observed
emission \PL\ \PS\ depends strongly on three factors that influence
the observation.  Firstly, it is known that many loops lie along the
line-of-sight in the corona \citep{???}.  Each of these may have
different emission measures, temperatures and densities.  The location
of energy deposition may not be the same for each loop \citep{???}.
This implies that attempts to work back from the observed emission to
the underlying energy deposition must properly account for these
line-of-sight effects.


Secondly, the amount of energy and the wavelengths at which it is
radiated is a highly nonlinear function of the energy deposited in
each event, depending on the existing plasma properties and the
location of the energy deposition in the loop structure.  This must
all be consistently modeled on each loop and summed over multple loops
to mimic the line-of-sight effects, as noted above.  

Much effort has
been \cite{???} EBTEL.

EBTEL , O'Dwyer paper???

Finally, the AIA instrument filter response must also be understood
and modeled in in order to understand which emission lines contribute
to the final observed image, and ultimately the \Fps\ for each
region.  The AIA filter response has been studied by
\cite{2010AA...521A..21O}  


These three factors will all have an effect on the inferred properties
of the energy deposition rate.  The observed \PL\ \PS\ here, coupled
with the emission event hypothesis and a modeling effort that takes
account of the factors above, offers a new way to constrain the energy
deposition rate in the solar corona.  Under the hypothesis, the energy
deposition rate can be explored through the distribution of the number
of events along the line-of-sight and their temporal dependence,
expressed through the parameters $\alpha_{E}$ and $\gamma$ in Equation
\ref{eqn:energydistrib}.



\section{Conclusions}\label{sec:conc}

Time series of AIA 171\AA\ and 193\AA\ images show approximately power
law like properties over all regions of the solar surface.  Detecting
narrow band oscillatory power should take in to account the presence
of the \PL\ like \PS.  The power spectra have been modelled using
three components: a \PL\ \PS, a lognormal component, and a constant
background.  It is hypothesized that the power-law component is due to
the sum along the line of sight of a large number of statistically
similar energy deposition events.  It is also hypothesized that the
lognormal component is due a combination of the channeling of
oscillatory power from lower to higher levels of the solar atmosphere
and lower temperature emission lines in the AIA 171\AA\ and
193\AA\ bandpasses.

Future studies will include examining power spectra at many more
locations on and off the disk, as well as using the AIA 355\AA, 94\AA,
131\AA and 211\AA\ wavebands.  The power-law index in individual
pixels will also be determined, and correlated with other physical
quantities such as the emission from those pixels.  Finally, a better
understanding of the source of the power-law power spectrum will come
from comparing the \Fps\ of simulated observations to the \Fps\ of the
data in all the AIA wave-bands that are sensitive to higher
temperatures.  This will bring new constraints on the simulations and
improve our understanding of the temporal behavior of the solar
corona.


%% If you wish to include an acknowledgments section in your paper,
%% separate it off from the body of the text using the \acknowledgments
%% command.

%% Included in this acknowledgments section are examples of the
%% AASTeX hypertext markup commands. Use \url without the optional [HREF]
%% argument when you want to print the url directly in the text. Otherwise,
%% use either \url or \anchor, with the HREF as the first argument and the
%% text to be printed in the second.

\acknowledgments

We are grateful to the developers of SSWIDL
\citep{1998SoPh..182..497F}, IPython \citep{ipython}, SunPy
\citep{mumford-proc-scipy-2013}, PyMC
\citep{Patil:Huard:Fonnesbeck:2010:JSSOBK:v35i04}, matplotlib
\citep{Hunter:2007} and Scientific Python stack for providing data
preparation, manipulation, analysis and display packages.  This work
was supported by NASA award NNX13AE03G S01 funded through NASA ROSES
NNH12ZDA001N-SHP.


%% To help institutions obtain information on the effectiveness of their
%% telescopes, the AAS Journals has created a group of keywords for telescope
%% facilities. A common set of keywords will make these types of searches
%% significantly easier and more accurate. In addition, they will also be
%% useful in linking papers together which utilize the same telescopes
%% within the framework of the National Virtual Observatory.
%% See the AASTeX Web site at http://aastex.aas.org/
%% for information on obtaining the facility keywords.

%% After the acknowledgments section, use the following syntax and the
%% \facility{} macro to list the keywords of facilities used in the research
%% for the paper.  Each keyword will be checked against the master list during
%% copy editing.  Individual instruments or configurations can be provided 
%% in parentheses, after the keyword, but they will not be verified.

{\it Facilities:} \facility{SDO (AIA)}.

%% Appendix material should be preceded with a single \appendix command.
%% There should be a \section command for each appendix. Mark appendix
%% subsections with the same markup you use in the main body of the paper.

%% Each Appendix (indicated with \section) will be lettered A, B, C, etc.
%% The equation counter will reset when it encounters the \appendix
%% command and will number appendix equations (A1), (A2), etc.

\appendix

\section{Fitting models to the \mFps}\label{sec:app:ind}
In order to perform the fit, the noise in the \mFps\ must be
estimated.  To do that, the power distributions of Section
\ref{sec:anal} are considered.  The Fourier power at a given pixel is
not independent of the Fourier power at another pixel.  This is
because neighbouring pixels are highly likely to be physically
connected to each other, and so time-series in neighboring pixels are
likely to be strongly correlated (assuming that the background
structure exists over at least two pixels).  Also, the point spread
function of AIA spreads emission in one pixel over on to its
neighbors.  Secondly, the distributions are only approximately normal.

Let the standard error in the mean of samples from a distribution be
given by $\sigma_{mean}$.  It is defined as the standard deviation of
the sampling distribution of the mean:
\begin{equation}
\sigma_{mean} = \sigma / N
\label{eqn:sigmamean}
\end{equation}
where $\sigma$ is the standard deviation of the original distribution
and $N$ is the sample size.  This formula assumes that the
measurements taken are all independent of each other, which is not the
case for the Fourier power spectra studied here.  Hence the value of
$N$ cannot be set to the number of pixels in a region.

Hence, the number of effectively independent pixels time-series must
be calculated.  First, a pair of randomly chosen next-nearest neighbor
pixels are selected from the region.  Next nearest neighbor pixels are
chosen as close physical proximity is most likely to express the
strongest interdependence.  Next, the cross correlation $c(\tau)$
function between two randomly chosen next-nearest neighbor pixels is
calculated.  The cross correlation coefficient measures the linear
dependence of two time series $X(t)$ and $Y(t \pm \tau)$ at different
lags $\tau$.  This is used to define a coefficient of independence
between the neighbors,
\begin{equation}
\label{eqn:ind}
\rho = 1 - \max|c(\tau)|.
\end{equation}
When $\rho$ is 1, the cross-correlation coefficient is zero, and there
is no measurable linear dependence between neighboring time series.
When $\rho$ is zero, the cross correlation coefficient is $\pm 1$, and
the two time-series are linearly dependent on each other.  This
procedure is repeated 10,000 times and the mean value $\hat{\rho}$ is
calculated. The effective number of independent pixels in region is
given by
\begin{equation}
\label{eqn:nind}
N_{eff}= 1 + \hat{\rho}(N_{pixel}-1).
\end{equation}
The number of independent measurements $N$ is set to $N_{eff}$ in
Equation \ref{eqn:sigmamean}, and that value of $\sigma_{mean}$ is
used as the estimate of the error in the \Fps.  Table \ref{tab:neff}
shows the number of pixels in the region and the estimated number of
effective pixels.
\begin{deluxetable}{cccc}
\tabletypesize{\scriptsize} 
\tablecolumns{4}
\tablewidth{0pt}
\tablecaption{Number of pixels and
  effectively independent pixels as a function of region and waveband,
  as calculated following the procedure of Section
  \protect\ref{sec:app:ind}. \label{tab:neff}}
\tablehead{
\colhead{Region} &
\colhead{$N_{pixel}$} &
\colhead{Waveband} &
\colhead{$N_{eff}$} \\
}
\startdata
loop footpoints & 2000 & 171\AA & 183 \\
                &     & 193\AA & 185 \\
moss            & 2100 & 171\AA & 629 \\
                &     & 193\AA & 521 \\
quiet Sun       & 2500 & 171\AA & 598 \\
                &     & 193\AA & 639 \\
sunspot         & 7500 & 171\AA & 1505 \\
                &     & 193\AA & 1531 \\
\enddata
\end{deluxetable}



\subsection{Model fitting and selection}
The \mFpa\ are assumed to be normally distributed with a width given
by $\sigma_{mean}$ as calculated above.  Equations \ref{eqn:pwrlaw}
and \ref{eqn:pwrlawbump} are fit to the geometric mean power spectra
shown in Figures \ref{fig:fit171193}.  A five step fitting process is
used to fit Models 1 and 2 to the data. The maximum likelihood
function is constructed using PyMC version 2.3
\citep{Patil:Huard:Fonnesbeck:2010:JSSOBK:v35i04}. Maximum likelihood
estimates are found using SciPy version 0.13.2 \citep{scipy,
  scientificpython} via Powell's minimization method
\citep{Powell01011964, wpress86:numerical}.


\begin{enumerate}
\item Maximum likelihood estimates to the values of the parameters of
  Model 1 are found using the distribution widths $\sigma$.  Since
  these widths are much larger than $\sigma_{mean}$ is much wider than
  the standard error in the mean, it allows the fitting algorithm to
  converge to an approximate fit solution.
\item The estimated values to the Model 1 parameters are then used as
  initial guesses to the second estimate of the Model 1
  parameters. Maximum likelihood estimates to the values of the
  parameters of Model 1 are found using the distribution widths
  $\sigma_{mean}$, seeded with estimated parameter values from step 1.
  This generates a better estimate to the parameters of the model
  fit.  The resulting parameter estimates are the final estimates for
  Model 1.
\item The residuals of the model 1 fit to the data are then used to
  estimate the model parameters of $G(\nu)$.
\item Model 2 is fit to the data using the distribution widths
  $\sigma$, seeded with the maximum likelihood parameter values of
  Model 1 from step 2 and the estimated $G(\nu)$ model parameters from
  step 3.
\item Model 2 is fit to the data using the maximum likelihood Model 2
  parameter values at Step 4 and the widths $\sigma_{mean}$.  The
  resulting parameter estimates are the final estimates for Model 2.
\end{enumerate}

Models are selected using the Akaike and Bayesian Information Criteria
\cite{2007MNRAS.377L..74L} calculated using the final parameter
estimates of Models 1 and 2.



%% The reference list follows the main body and any appendices.
%% Use LaTeX's thebibliography environment to mark up your reference list.
%% Note \begin{thebibliography} is followed by an empty set of
%% curly braces.  If you forget this, LaTeX will generate the error
%% "Perhaps a missing \item?".
%%
%% thebibliography produces citations in the text using \bibitem-\cite
%% cross-referencing. Each reference is preceded by a
%% \bibitem command that defines in curly braces the KEY that corresponds
%% to the KEY in the \cite commands (see the first section above).
%% Make sure that you provide a unique KEY for every \bibitem or else the
%% paper will not LaTeX. The square brackets should contain
%% the citation text that LaTeX will insert in
%% place of the \cite commands.

%% We have used macros to produce journal name abbreviations.
%% AASTeX provides a number of these for the more frequently-cited journals.
%% See the Author Guide for a list of them.

%% Note that the style of the \bibitem labels (in []) is slightly
%% different from previous examples.  The natbib system solves a host
%% of citation expression problems, but it is necessary to clearly
%% delimit the year from the author name used in the citation.
%% See the natbib documentation for more details and options.

\bibliography{references}
\end{document}


However, other more complex models are certainly possible.  For
example, \cite{1993ASPC...42..111H} modelled the background
chromospheric Fourier \PS\ of a series of Ca-II K-line
images using a 19-parameter fit consisting of a constant background,
three components of the form
\begin{equation}
\label{eqn:harvey1}
P(\nu) \propto \frac{1}{1 + (2\pi \nu T_{0})^{b}}.
\end{equation}
and two components of the form
\begin{equation}
\label{eqn:harvey2}
P(\nu) \propto
\left(
\frac{\nu}{\nu_{0}}
\right)^{c}
\left(
\frac{\Gamma^{2}}{(\nu-\nu_{0})^{2} + \Gamma^{2}}
\right)^{b}.
\end{equation}


Finally, only two models of the Fourier \PS\ have been
considered.  The choice of equation \ref{eqn:pwrlaw} is motivated by
the observations of all four regions considered.  Equation
\ref{eqn:pwrlawbump} was motivated by previous observations of
oscillatory content in the solar atmosphere.  Equation
\ref{eqn:pwrlawbump} is shown to be the preferred model in all cases.


The mean power spect
