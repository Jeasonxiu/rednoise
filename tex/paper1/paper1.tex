%%
%% Beginning of file 'sample.tex'
%%
%% Modified 2005 December 5
%%
%% This is a sample manuscript marked up using the
%% AASTeX v5.x LaTeX 2e macros.

%% The first piece of markup in an AASTeX v5.x document
%% is the \documentclass command. LaTeX will ignore
%% any data that comes before this command.

%% The command below calls the preprint style
%% which will produce a one-column, single-spaced document.
%% Examples of commands for other substyles follow. Use
%% whichever is most appropriate for your purposes.
%%
%%\documentclass[12pt,preprint]{aastex}

%% manuscript produces a one-column, double-spaced document:


\documentclass[preprint2]{aastex}
\usepackage{graphicx}
\usepackage{color}
\usepackage{multirow}
%\usepackage{showkeys}
\usepackage{epsfig}
\usepackage{natbib}
\usepackage{wasysym}
%\usepackage{subfigure}
\usepackage{verbatim}

\bibliographystyle{apj}	%% AASTeX

%% preprint2 produces a double-column, single-spaced document:

%% \documentclass[preprint2]{aastex}

%% Sometimes a paper's abstract is too long to fit on the
%% title page in preprint2 mode. When that is the case,
%% use the longabstract style option.

%% \documentclass[preprint2,longabstract]{aastex}

%% If you want to create your own macros, you can do so
%% using \newcommand. Your macros should appear before
%% the \begin{document} command.
%%
%% If you are submitting to a journal that translates manuscripts
%% into SGML, you need to follow certain guidelines when preparing
%% your macros. See the AASTeX v5.x Author Guide
%% for information.

\newcommand{\vdag}{(v)^\dagger}
\newcommand{\myemail}{jack.ireland@nasa.gov}
\newcommand{\PS}{power spectrum}
\newcommand{\PL}{power-law}
\newcommand{\Fps}{Fourier \PS}

%% You can insert a short comment on the title page using the command below.

%\slugcomment{Not to appear in Nonlearned J., 45.}

%% If you wish, you may supply running head information, although
%% this information may be modified by the editorial offices.
%% The left head contains a list of authors,
%% usually a maximum of three (otherwise use et al.).  The right
%% head is a modified title of up to roughly 44 characters.
%% Running heads will not print in the manuscript style.

\shorttitle{Properties of coronal Fourier power spectra}
\shortauthors{Ireland et al.}

%% This is the end of the preamble.  Indicate the beginning of the
%% paper itself with \begin{document}.

\begin{document}

%% LaTeX will automatically break titles if they run longer than
%% one line. However, you may use \\ to force a line break if
%% you desire.

\title{Properties of coronal Fourier power spectra and their
  implications for coronal seismology and solar atmospheric heating.}

%% Use \author, \affil, and the \and command to format
%% author and affiliation information.
%% Note that \email has replaced the old \authoremail command
%% from AASTeX v4.0. You can use \email to mark an email address
%% anywhere in the paper, not just in the front matter.
%% As in the title, use \\ to force line breaks.

\author{J. Ireland}
\affil{ADNET Systems, Inc.,NASA Goddard Space Flight Center, MC. 671.1 Greenbelt, MD
  20771, USA.}

\author{R. T. J. McAteer}
\affil{Department of Astronomy, New Mexico State University, Las
  Cruces, NM.}

\author{A. R. Inglis}
\affil{Catholic University of America / NASA Goddard Space Flight Center, MC. 671.1 Greenbelt, MD
  20771, USA.}

%% Mark off your abstract in the ``abstract'' environment. In the manuscript
%% style, abstract will output a Received/Accepted line after the
%% title and affiliation information. No date will appear since the author
%% does not have this information. The dates will be filled in by the
%% editorial office after submission.

\begin{abstract}
The Fourier power spectra of regions of the solar corona are
investigated using AIA 171\AA\ and 193\AA\ data.  It is shown that the
Fourier power spectra in all the regions show a power law behaviour
approximately.  This suggests that in many parts of the corona a
background model consisting of white noise and a trend cannot be
assumed when analyzing for the presence of oscillations in the corona.
Further, it is shown that the observed power law power spectrum can be
formed by summing a distribution of exponentially decaying emission
events along the line of sight.  This is consistent with the notion
that the solar atmosphere is heated everywhere by small energy
deposition events.
\end{abstract}

%% Keywords should appear after the \end{abstract} command. The uncommented
%% example has been keyed in ApJ style. See the instructions to authors
%% for the journal to which you are submitting your paper to determine
%% what keyword punctuation is appropriate.

\keywords{???}

%% From the front matter, we move on to the body of the paper.
%% In the first two sections, notice the use of the natbib \citep
%% and \citet commands to identify citations.  The citations are
%% tied to the reference list via symbolic KEYs. The KEY corresponds
%% to the KEY in the \bibitem in the reference list below. We have
%% chosen the first three characters of the first author's name plus
%% the last two numeral of the year of publication as our KEY for
%% each reference.


%% Authors who wish to have the most important objects in their paper
%% linked in the electronic edition to a data center may do so by tagging
%% their objects with \objectname{} or \object{}.  Each macro takes the
%% object name as its required argument. The optional, square-bracket 
%% argument should be used in cases where the data center identification
%% differs from what is to be printed in the paper.  The text appearing 
%% in curly braces is what will appear in print in the published paper. 
%% If the object name is recognized by the data centers, it will be linked
%% in the electronic edition to the object data available at the data centers  
%%
%% Note that for sources with brackets in their names, e.g. [WEG2004] 14h-090,
%% the brackets must be escaped with backslashes when used in the first
%% square-bracket argument, for instance, \object[\[WEG2004\] 14h-090]{90}).
%%  Otherwise, LaTeX will issue an error. 

\section{Introduction}

The solar corona is observed to support many different oscillatory
phenomena.



\section{Observations}\label{sec:obs}

Data was obtained from SDO-AIA using the Lockheed Martin Solar
Astrophysics cutout service (http://lmsal.com/get\_aia\_data/).  Six
hours of image data at 12 second cadence was selected in the 171\AA
and 193\AA\ wavebands (2012-09-23 00:00 - 06:00 UT, 1800 images per
waveband).  These wavebands were selected in order to obtain a wide
range of approximate temperatures in the solar corona.

The downloaded cutout data was prepared for analysis using the
SolarSoft / IDL routines READ\_SDO and AIA\_PREP.  These procedures
move the downloaded level 1.0 FITS files to level 1.5 FITS files.  All
further processing from this point was performed in the SunPy 0.4
(http://www.sunpy.org) data analysis environment.  Images were
de-rotated to compensate for solar rotation, based on the center of
the field of view of each image.  After de-rotation, each image layer
was co-registered with the image half way through the dataset, at
approximately 03:00 UT.

\section{Analysis}\label{sec:anal}
We wish to analyze the frequency content of these resulting datacubes
as a function of spatial location.  Four locations were chosen in the
data representing four different types of physical locations in the
solar atmosphere.  These were a quiet Sun region, a region that lies
on top of a sunspot, a footpoint region, and a coronal moss region.
Figures \ref{fig:loc171193} show the location of the regions on sample
images from the time range considered.

\begin{figure}
\plottwo{shutdownfun3_6hr_disk_1.5_171.eps}{shutdownfun3_6hr_disk_1.5_193.eps}
\caption{Locations of the regions analyzed in each of AIA 171\AA\ and
  193\AA, overlaid on sample images taken halfway through the six hour
  span used to form the time-series analyzed.}
\label{fig:loc171193}
\end{figure}

Consider the emission from a single pixel, $I(t)$.  The Fourier power
spectrum for this emission was calculated as follows.  First, the
time-series was normalized by calculating $(I(t) - <I(t)>)/<I(t)>$.
This normalizes the emission.  Secondly, the normalized time-series
was apodized using the Hanning window, reducing ringing effects.  This
results in a time-series from which its Fourier power spectrum was
analyzed. 

Figures \ref{fig:compare171193} show example power spectra, plotted using
a log-log scale, from a single location inside each of the regions
shown in Figure \ref{fig:loc171193}.  The shape of
the power spectra indicate that the Fourier power approximately
follows a power law at lower frequencies and flattens out at higher
frequencies.

\begin{figure}
\plottwo{loopfootpoints.centralpower.eps}{qs.centralpower.eps}
\plottwo{sunspot.centralpower.eps}{moss.centralpower.eps}
\caption{Log-log plot of the Fourier power of a time-series at the AIA
  pixel overlying a randomly chosen pixel in each of the regions
  considered (see Figure \protect\ref{fig:loc171193}).  The vertical
  black lines indicate the 3 and 5 minute frequencies.}
\label{fig:compare171193}
\end{figure}


Figures \ref{fig:dist171} and \ref{fig:dist193} show the distributions
of the logarithm of the Fourier power for certain frequencies in the
spectrum.  The distributions all approximately follow a normal
distribution.  This indicates that the power in each each at each
frequency is approximately log-normally distributed.  This observation
guides the choice of how to average the power spectrum over each of
the selected physical regions.  Since the Fourier power is
log-normally distributed, the arithmetic mean of the Fourier power is
biased by the very largest powers.  Hence the mean of the logarithm of
the Fourier power is considered, which is equivalent to the geometric
mean of the Fourier power.

\begin{figure}
\plottwo{shutdownfun3_6hr_disk_1.5_171_loopfootpoints.hanning.relative.power_spectra_distributions.eps}{shutdownfun3_6hr_disk_1.5_171_moss.hanning.relative.power_spectra_distributions.eps}
\plottwo{shutdownfun3_6hr_disk_1.5_171_qs.hanning.relative.power_spectra_distributions.eps}{shutdownfun3_6hr_disk_1.5_171_sunspot.hanning.relative.power_spectra_distributions.eps}
\caption{Distribution of Fourier power at some sample frequencies (see also Figure \ref{fig:dist193} for the corresponding AIA 171\AA\ plots.)}
\label{fig:dist171}
\end{figure}

\begin{figure}
\plottwo{shutdownfun3_6hr_disk_1.5_193_loopfootpoints.hanning.relative.power_spectra_distributions.eps}{shutdownfun3_6hr_disk_1.5_193_moss.hanning.relative.power_spectra_distributions.eps}
\plottwo{shutdownfun3_6hr_disk_1.5_193_qs.hanning.relative.power_spectra_distributions.eps}{shutdownfun3_6hr_disk_1.5_193_sunspot.hanning.relative.power_spectra_distributions.eps}
\caption{Distribution of Fourier power at some sample frequencies (see
  also Figure \ref{fig:dist171} for the corresponding AIA
  171\AA\ plots.)}
\label{fig:dist193}
\end{figure}

These geometric mean Fourier power spectra are shown in Figures
\ref{fig:fit171} and \ref{fig:fit193}, for each of the physical
regions considered.  It is clear that, in general, the Fourier power
spectra exhibit power-law like characteristics.

\begin{figure}
\plottwo{shutdownfun3_6hr_disk_1.5_171_loopfootpoints.hanning.relative.logiobs.1.model_fit_compare.pymc.eps}{shutdownfun3_6hr_disk_1.5_171_moss.hanning.relative.logiobs.1.model_fit_compare.pymc.eps}
\plottwo{shutdownfun3_6hr_disk_1.5_171_qs.hanning.relative.logiobs.1.model_fit_compare.pymc.eps}{shutdownfun3_6hr_disk_1.5_171_sunspot.hanning.relative.logiobs.1.model_fit_compare.pymc.eps}
\caption{Geometric mean of the Fourier power spectra in each of the
  four regions studied in AIA 171\AA\ (shown in black).  Also shown
  are the posterior mean and maximum likelihood fits for model $M_{1}$
  and $M_{2}$ (see Section \ref{sec:anal} and Appendix
  \ref{sec:app:ind} for more detail).  Model fit parameter values and
  their 68\% credible intervals are quoted in Table
  \ref{tab:parameters}. See also Figure \ref{fig:fit193} for the
  corresponding AIA 193\AA\ plots.}
\label{fig:fit171}
\end{figure}

\begin{figure}
\plottwo{shutdownfun3_6hr_disk_1.5_193_loopfootpoints.hanning.relative.logiobs.1.model_fit_compare.pymc.eps}{shutdownfun3_6hr_disk_1.5_193_moss.hanning.relative.logiobs.1.model_fit_compare.pymc.eps}
\plottwo{shutdownfun3_6hr_disk_1.5_193_qs.hanning.relative.logiobs.1.model_fit_compare.pymc.eps}{shutdownfun3_6hr_disk_1.5_193_sunspot.hanning.relative.logiobs.1.model_fit_compare.pymc.eps}
\caption{Geometric mean of the Fourier power spectra in each of the
  four regions studied in AIA 193\AA\ (shown in black).  Also shown
  are the posterior mean and maximum likelihood fits for model $M_{1}$
  and $M_{2}$ (see Section \ref{sec:anal} and Appendix
  \ref{sec:app:ind} for more detail).  Model fit parameter values and
  their 68\% credible intervals are quoted in Table
  \ref{tab:parameters}. See also Figure \ref{fig:fit171} for the
  corresponding AIA 171\AA\ plots.}
\label{fig:fit193}
\end{figure}

\begin{deluxetable}{cccccccc}
\tabletypesize{\scriptsize}
\tablecolumns{5}
\tablewidth{0pt}
\tablecaption{Parameter values and uncertainty estimates derived for
    model \protect$M_{2}$.\label{tab:parameters}}
\tablehead{
\colhead{Region} & 
\colhead{Waveband}    & 
\multicolumn{3}{c}{power law} & \multicolumn{3}{c}{Gaussian} \\
\colhead{}          & 
\colhead{}          & 
\colhead{$\log_{10}A$}          & 
\colhead{$n$}      & 
\colhead{$\log_{10}C$}      &
\colhead{$\log_{10}\alpha$} &
\colhead{$\beta$}      &
\colhead{$\sigma$}\\
\colhead{}          & 
\colhead{}          & 
\colhead{}          & 
\colhead{}      & 
\colhead{}      &
\colhead{} &
\colhead{(mHz)}      &
\colhead{(decades of frequency)}\\
}
\startdata
loop footpoints & 171\AA  & $0.65^{+0.02}_{-0.02}$  &$2.27^{+0.02}_{-0.02}$  & $-4.15^{+0.00}_{-0.00}$  & $-2.16^{+0.12}_{-0.11}$ & $0.64^{+0.10}_{-0.10}$  & $0.33^{+0.02}_{-0.02}$  \\
                & 193\AA  & $0.21^{+0.02}_{-0.02}$  & $2.27^{+0.03}_{-0.03}$ & $-4.26^{+0.00}_{-0.00}$  & $-2.24^{+0.09}_{-0.10}$ & $0.37^{+0.06}_{-0.04}$  & $0.40^{+0.01}_{-0.02}$  \\
                & & & & & & & \\

moss            & 171\AA  & $0.47^{+0.01}_{-0.01}$  & $1.86^{+0.01}_{-0.01}$ & $-3.97^{+0.01}_{-0.00}$  & $-1.33^{+0.02}_{-0.02}$  & $1.05^{+0.03}_{-0.02}$  & $0.31^{+0.00}_{-0.01}$  \\
                & 193\AA  & $0.57^{+0.01}_{-0.01}$  & $2.09^{+0.01}_{-0.01}$ & $-4.28^{+0.00}_{-0.00}$  & $-1.97^{+0.02}_{-0.03}$  & $1.43^{+0.03}_{-0.03}$  & $0.25^{+0.00}_{-0.00}$  \\
                & & & & & & & \\

quiet Sun       & 171\AA  & $0.67^{+0.01}_{-0.01}$  & $1.73^{+0.01}_{-0.01}$ & $-2.86^{+0.00}_{-0.00}$  & $-3.49^{+0.04}_{-0.04}$  & $4.50^{+0.14}_{-0.09}$  & $0.10^{+0.01}_{-0.01}$  \\
                & 193\AA  & $0.06^{+0.01}_{-0.01}$  & $2.25^{+0.01}_{-0.01}$ & $-3.38^{+0.00}_{-0.00}$  & $-4.54^{+0.06}_{-0.10}$  & $4.72^{-0.17}_{-0.16}$  & $0.13^{+0.01}_{-0.02}$  \\
                & & & & & & & \\

sunspot         & 171\AA  & $0.51^{+0.01}_{-0.01}$  & $2.27^{+0.02}_{-0.02}$ & $-3.79^{+0.00}_{-0.00}$  & $-2.94^{+0.02}_{-0.02}$  & $2.18^{+0.08}_{-0.07}$  & $0.29^{+0.01}_{-0.01}$  \\
                & 193\AA  & $0.02^{+0.02}_{-0.01}$  & $2.19^{+0.02}_{-0.04}$ & $-3.85^{+0.00}_{-0.00}$  & $-3.17^{+0.04}_{-0.04}$  & $1.87^{+0.11}_{-0.11}$  & $0.35^{+0.01}_{-0.01}$  \\
\enddata
%% Text for table notes should follow after the \enddata but before
%% the \end{deluxetable}. Make sure there is at least one \tablenotemark
%% in the table for each \tablenotetext.
\end{deluxetable}

The results for the quiet Sun, loop footpoints and sunspot regions
suggest a power-law power spectrum.  This can be modeled as
\begin{equation}
\label{eqn:pwrlaw}
\mbox{Model $M_{1}$}: P_{1}(\nu) = A\nu^{-n} + C,
\end{equation}
where $\nu$ is the normalized frequency, $A>0$ is a proportionality
constant and $n>0$ is the power law index.  The quantity $C>0$ models
the high-frequency / low power end of the spectrum where the detection
properties of the detector apparatus are assumed to dominate the
observations.

The moss results for 171\AA\ and 193\AA\ are notably different from
the other results.  In comparison to the general trend observed in
other regions, there appears to be excess Fourier power in the range
1-10 mHz.  Previous studies (???) have shown that the moss is a source
of Fourier power.  This suggests that a second model $M_{2}$ for these
regions should be considered.  The second model $M_{2}$ has two
contributions to the overall \Fps; a background power-law of the form
of $P_{1}$, and a contribution that is more localized, $G(\nu)$.  The
model is given by
\begin{equation}
\label{eqn:pwrlawbump}
\mbox{Model $M_{2}$}: P_{2}(\nu) = P_{1}(\nu) + G(\nu)
\end{equation}
where the localized contribution is described by
\begin{equation}
\label{eqn:bump}
G(\nu) = \alpha\exp\left[-\frac{(\ln(\nu)-\beta)^{2}}{2\sigma^{2}}\right].
\end{equation}
Both these models are fit to the geometrically averaged Fourier power
spectrum in both wavelengths and in all four regions.  Figures
\ref{fig:fit171} and \ref{fig:fit193} show the fit results of
each model to the data.  Table \ref{tab:parameters} shows the fitted
parameter values and an error estimate for model $M_{2}$.  For details
on the fitting procedure, see Section \ref{sec:app:ind}.

\section{Discussion}
The presence of a power law Fourier spectrum in coronal emission poses
questions about how that emission is formed.  The power law power
spectrum also has implications for the detection of narrow frequency
band oscillations against such a background emission, which is the
purview of coronal seismology.  We discuss these two topics in the
sections below.


\subsection{Effect of background power spectrum assumptions on the
  detection of oscillatory power}
\label{ssec:corseis}

Section \ref{sec:anal} demonstrates the presence of a red
noise-like signal in two of the most commonly used wavebands for
coronal seismology, SDO-AIA 171\AA\ and 193\AA.  The observed spectra
generally show a power law-like behavior at time-scales of interest in
coronal seismology, and longer.  Within the confines of the models we
have chosen, coronal moss regions show the largest contribution from a
non-power law power spectrum source over a relatively limited range of
frequencies.  The frequencies at which the Gaussian contribution is
larger than the underlying power-law distribution could be used to
define an average frequency range within which the majority
contribution to the overall Fourier power is not associated with the
underlying power-law distribution.  However, the \Fps\ in this range
would still have to analyzed using both the power-law and Gaussian
contributions to the overall power, and better estimates to the
power-law parameters will be obtained if the full frequency range of
the observations are used.

The presence of a non-power law contribution to the \Fps\ does not
change the fact that the background emission is essentially power-law
like.  For the frequencies that have been of interest in coronal
seismology, the background power spectrum is therefore scale free on
average.  If there is no time-scale that can be used to remove a
background trend, then the time-series has to be analyzed without this
processing step.  A lack of awareness of the background red-noise
nature of the emission can cause problems in attempting to analyze for
the presence of narrow frequency band oscillations, as the following
example demonstrates.

\begin{figure}
\epsscale{.80}
\plotone{white_red_compare.eps}
\caption{Comparison of the effect of assuming a white or red noise
  background on the detection of an oscillatory signal for simulated
  time-series.  Plot (a) shows the simulated time-series, generated
  from a power spectrum \protect$P(f)\approx f^{-1.77}$ and no
  explicit oscillation included.  Plot (b) shows the wavelet power
  spectrum with the cone-of-influence (shaded area) and regions above
  the 95\% confidence level, assuming a white-noise (Gaussian)
 background .  Plot (c) shows the global wavelet power spectrum for
  this wavelet transform.  Plots (d) and (e) are the same as plots (b)
  and (c) under the assumption of a red-noise background.}
\label{fig:comparison}
\end{figure}

Figure \ref{fig:comparison} shows how a red-noise power spectrum can
be mistakenly thought to contain an oscillatory signal.  The time
series (Figure \ref{fig:comparison}(a)) is constructed from a power
spectrum $P(f)\approx f^{-1.77}$ with no explicit oscillatory content,
following the construction procedure of \cite{2010MNRAS.402..307V}.
Figure \ref{fig:comparison}(b) shows that the assumption of a
white-noise background and a 95\% confidence level leads to the
positive detection of significant oscillatory power in this time
series. Figure \ref{fig:comparison}(c) shows that the assumption of a
red-noise background with the same confidence level significantly
reduces the area for which a detection may be claimed.  This simulated
data, along with the results of Section \ref{sec:anal}, suggest that
when examining the wavelet transforms of time-series for wave packets,
a background red-noise power spectrum should be assumed, along with
higher confidence levels, in order to minimize the effects of
mistakenly identifying red-noise as evidence for an oscillatory
signal.

Even this approach to identifying oscillatory power is not quite
complete, as it simply rejects the null hypothesis that the power
spectrum can be adequately explained by a red-noise power spectrum.
Model comparison similar to that described in Section \ref{sec:anal}
is required to prefer a model with oscillatory content over one
without. The difficulty in finding narrow-band features in Fourier
power spectra against a red-noise background power spectrum has been
recognized by \cite{2010MNRAS.402..307V} in the study of XMM-Newton
observations of highly variable Seyfert 1 galaxies.
\cite{2010MNRAS.402..307V} presents a model comparison and fitting
technique is directly applicable to the search for oscillatory power
in coronal time-series.


\subsection{Implications for coronal seismology and automated
  oscillation detection algorithms}\label{sec:oscdetect}

The discussion of Section \ref{ssec:corseis} shows that red-noise
power spectra has a powerful effect on the claims of a detection of
narrow-frequency band oscillations in time-series.  When looking at
extended regions of coronal imaging data, evidence for the presence of
an oscillation in the corona is strengthened if it can be shown that
narrow frequency-band oscillations occur significantly above a
red-noise background signal in several neighboring pixels.  Coronal
seismology derives its observational evidence from such detections,
and automated detection algorithms hold out the hope of greatly
increasing the number of oscillating regions detected.

Figure \ref{fig:compare171193} shows that time series from individual
pixels have power-law like power spectra.  This has implications for
the design of automated oscillation detection algorithms.  Such an
algorithm must somehow take into account the nature of the power
spectrum.  The power-law power spectrum itself lacks an easily
definable time-scale at or below which a background trend can be
defined.  Automated detection algorithns that rely on background trend
subtraction such as \cite{2010SoPh..264..403I} must therefore look to
other reasons for to justify a time-scale.

As has been shown above, the assumption of a Gaussian-distributed
noise when red noise power-spectrum is actually present, coupled with
too low a confidence level, can lead to misleading results when
looking for oscillatory signals.  Many detection algorithms do make
this assumption, for example \cite{2004SoPh..223....1D},
\cite{2007SoPh..241..397N}, \cite{2008SoPh..248..395S} and
\cite{2010SoPh..264..403I}.  Such approaches have yet to be tested
assuming a red-noise background power spectrum, and so their efficacy
is unknown.  \cite{2004SoPh..223....1D} and \cite{2008SoPh..248..395S}
are wavelet-based approaches, and Figure \ref{fig:comparison} does
suggest that a red-noise background spectrum will make a significant
difference. 

All the automated detection algorithms described above attempt to find
individual pixels that are obviously oscillating at some frequency,
using the strength of the oscillation at that frequency (compared to
some assumption of the background) as the key quantity to measure.
These pixels are then clumped together and the quality of the signal
is then assessed.  Detection algorithms based on wave coherence, such
as \cite{2008SoPh..252..321M}, implement wave detection through
finding spatially contiguous regions of the corona which are coherent
in a given frequency range.  The frequency range searched is based on
previous experience.  Oscillations that are detected propagating along
loops are often found by considering distance-time plots
\citep{2000AA...355L..23D, 2003AA...404L...1K} and looking for alternating
brightening/darkening in those plots as a indicator of a propagating
wave.  This is essentially a coherence-based approach, as the user is
assessing the likelihood that an oscillation has been detected based
on the observed signal in multiple contiguous pixels.  However, each
pixel in these plots probably has a red-noise background power
spectrum, and so the discussion of Section \ref{ssec:corseis} suggests
that the significance of any claimed oscillation is much reduced.


\subsection{``Excess'' emission}\label{ssec:excess}
The models fits of Section \ref{sec:anal} indicate that equation
\ref{eqn:pwrlaw} is preferred over equation \ref{eqn:pwrlawbump} as the
superior description of the observed Fourier power spectrum.  This
model is consistent with previous observations by TRACE of power in
the frequency range of 1 - 10 mHz.  The moss region shows the largest
presence of this excess emission.

It would be tempting to take this excess contribution and infer a
distribution of field line angles with the solar surface.  This
distribution relies on the assumption that all the
Gaussian-distributed component is entirely due to the channeling of
photospheric oscillatory power up to ``coronal'' heights and
temperatures.


The AIA 171\AA\ and 193\AA\ wavebands are relatively wide, and include
contributions from plasma at many different temperatures, heights in
the atmosphere, and many different elements \cite{}.  A deeper
understanding of the contributions from these different plasmas is
required before a good estimate of the distribution of the angle
coronal loop footpoints make with the surface can be made.

Finally, only two models of the Fourier power spectrum are considered.
The choice of equation \ref{eqn:pwrlawbump} is motivated by previous
observations.  However, other models are certainly possible.  For
example, \cite{1993ASPC...42..111H} modelled the


\subsection{Nature of the power law power spectrum}\label{ssec:nplps}
The first contribution to both models $P_{0}$ and $P_{i}$ is a
red-noise like power spectrum.  It is present in both wavebands
studied and in all the regions, and so parsimony suggests that the
mechanism of its creation is both similar and ubiquitous in all parts
of the solar atmosphere.  We hypothesize that the \PL\ \PS\ is due to
the sum of a distribution of a large number of energy deposition
events having different energies. \cite{2011soca.book.....A} describes
how a \PL\ \PS\ may obtained from such a distribution, and that
argument outlined below.

Each energy deposition event is modeled as an exponentially decaying
function of time $t$
\begin{equation}
\label{eqn:expdecay}
f(t) = \frac{E}{T}\exp\left(-\frac{t}{T}\right),
\end{equation}
for some timescale $T$ and energy $E$.  The corresponding \Fps\ is
\begin{equation}
\label{eqn:ftexpdecay}
P(\nu) = \frac{E}{1 + (2\pi \nu T)^{2}}.
\end{equation}
The total \Fps\ along the line of sight is given by
\begin{equation}
\label{eqn:sumftexpdecay}
P_{T}(\nu) = \sum_{T}N(T)P_{T}(\nu)
\end{equation}
where $N(T)$ is the distribution of time-scales for all the events
along the line of sight.  Further, if the number of events of a given
energy $E$ is assumed to be
\begin{equation}
\label{eqn:energydistrib}
N(E) \propto E^{-\alpha_{E}}
\end{equation}
and the energy in each event depends on its time scale $T$ such that
\begin{equation}
\label{eqn:energytime}
E \propto T^{1+\gamma}
\end{equation}
then it can be shown \cite{} that the observed power spectrum can be
approximated by
\begin{equation}
\label{eqn:finalfps}
P(\nu) \propto \nu^{(2-\alpha_{E})(1+\gamma)}.
\end{equation}

This derivation shows it is possible to generate \PL\ power spectra
using swarms of statistically similar events.  The detailed
application of this model of the observed background \PL\ power
spectra will depend strongly on three factors that influence the
observation.  Firstly, the distribution of events along the line of
sight and their temporal dependence (expressed through the parameters
$\alpha_{E}$ and $\gamma$ in Equation \ref{eqn:energydistrib}) have
strong effects on the observed \Fps.  Secondly, the energy radiated as
a function of the energy deposited in each event and the existing
plasma properties must be consistently modeled.  This is required to
understand how much radiation at each wavelength is emitted from the
plasma.  Finally, the AIA instrument filters the emitted radiation,
and this too must be modeled in order to understand the observed \Fps.
These three features must be modeled in order to test the hypothesis
that ubiquitous energy deposition events are the root cause of the
observed power-law like behavior of the Fourier power spectra.

There has been a considerable amount of work in modeling the
AIA-detected plasma response...  \cite{} 



\section{Conclusions}\label{sec:conc}

Comparing the \Fps\ of the simulated observations to the \Fps\ of the
data will bring new constraints on the simulations that will hopefully
improve our understanding of the temporal behavior of the solar corona.


%% If you wish to include an acknowledgments section in your paper,
%% separate it off from the body of the text using the \acknowledgments
%% command.

%% Included in this acknowledgments section are examples of the
%% AASTeX hypertext markup commands. Use \url without the optional [HREF]
%% argument when you want to print the url directly in the text. Otherwise,
%% use either \url or \anchor, with the HREF as the first argument and the
%% text to be printed in the second.

\acknowledgments

We are grateful to the developers of SSWIDL \cite{}, SunPy \cite{},
PyMC \cite{} and matplotlib \cite{} for providing data preparation,
manipulation, analysis and display packages.  This work was supported
by the NASA....

%% To help institutions obtain information on the effectiveness of their
%% telescopes, the AAS Journals has created a group of keywords for telescope
%% facilities. A common set of keywords will make these types of searches
%% significantly easier and more accurate. In addition, they will also be
%% useful in linking papers together which utilize the same telescopes
%% within the framework of the National Virtual Observatory.
%% See the AASTeX Web site at http://aastex.aas.org/
%% for information on obtaining the facility keywords.

%% After the acknowledgments section, use the following syntax and the
%% \facility{} macro to list the keywords of facilities used in the research
%% for the paper.  Each keyword will be checked against the master list during
%% copy editing.  Individual instruments or configurations can be provided 
%% in parentheses, after the keyword, but they will not be verified.

{\it Facilities:} \facility{SDO (AIA)}.

%% Appendix material should be preceded with a single \appendix command.
%% There should be a \section command for each appendix. Mark appendix
%% subsections with the same markup you use in the main body of the paper.

%% Each Appendix (indicated with \section) will be lettered A, B, C, etc.
%% The equation counter will reset when it encounters the \appendix
%% command and will number appendix equations (A1), (A2), etc.

\appendix

\section{Model fitting procedure}\label{sec:app:ind}
Equations \ref{eqn:pwrlaw} and \ref{eqn:pwrlawbump} are fit to the
geometric mean power spectra shown in Figures \ref{fig:fit171} and
\ref{fig:fit193}.  The frequency $\nu$ is defined as $f/f_{0}$ where
$f$ is the FFT frequency in Hz, and $f_{0}$ is the lowest non-zero FFT
frequency defined by the data (Section \ref{sec:obs}).

As noted in Section \ref{sec:anal}, the logarithm of the Fourier power for all
regions may be approximated by a normal distribution $N(\mu, \sigma)$.
The maximum likelihood estimate of the mean and variance of normal
distribution are given by
\begin{equation}
\hat{\mu} = \frac{1}{N_{pixel}}\sum_{i=1}^{N_{pixel}}X_{i}
\label{eqn:mlmean}
\end{equation}
and
\begin{equation}
\hat{\sigma^{2}} = \frac{1}{N_{pixel}}\sum_{i=1}^{N_{pixel}}(X_{i}-\hat{\mu})^{2}
\label{eqn:sigmamean}
\end{equation}
where $N_{pixel}$ is the number of pixels in each region.



The estimated
error in the estimated error is
\begin{equation}
\sqrt{\hat{\sigma^{2}}}/\sqrt{N}
\label{eqn:errorsigmamean}
\end{equation}






%% The reference list follows the main body and any appendices.
%% Use LaTeX's thebibliography environment to mark up your reference list.
%% Note \begin{thebibliography} is followed by an empty set of
%% curly braces.  If you forget this, LaTeX will generate the error
%% "Perhaps a missing \item?".
%%
%% thebibliography produces citations in the text using \bibitem-\cite
%% cross-referencing. Each reference is preceded by a
%% \bibitem command that defines in curly braces the KEY that corresponds
%% to the KEY in the \cite commands (see the first section above).
%% Make sure that you provide a unique KEY for every \bibitem or else the
%% paper will not LaTeX. The square brackets should contain
%% the citation text that LaTeX will insert in
%% place of the \cite commands.

%% We have used macros to produce journal name abbreviations.
%% AASTeX provides a number of these for the more frequently-cited journals.
%% See the Author Guide for a list of them.

%% Note that the style of the \bibitem labels (in []) is slightly
%% different from previous examples.  The natbib system solves a host
%% of citation expression problems, but it is necessary to clearly
%% delimit the year from the author name used in the citation.
%% See the natbib documentation for more details and options.

\bibliography{references}
\end{document}
