%%
%% Beginning of file 'sample.tex'
%%
%% Modified 2005 December 5
%%
%% This is a sample manuscript marked up using the
%% AASTeX v5.x LaTeX 2e macros.

%% The first piece of markup in an AASTeX v5.x document
%% is the \documentclass command. LaTeX will ignore
%% any data that comes before this command.

%% The command below calls the preprint style
%% which will produce a one-column, single-spaced document.
%% Examples of commands for other substyles follow. Use
%% whichever is most appropriate for your purposes.
%%
%%\documentclass[12pt,preprint]{aastex}

%% manuscript produces a one-column, double-spaced document:


\documentclass[onecolumn]{emulateapj}
%\documentclass[twocolumn]{aastex}
%\documentclass[referee]{aastex}
\usepackage{graphicx}
\usepackage{color}
\usepackage{multirow}
\usepackage{setspace}

%\usepackage{showkeys}
\usepackage{epsfig}
\usepackage{natbib}
\usepackage{wasysym}
%\usepackage{subfigure}
\usepackage{verbatim}

\bibliographystyle{apj}	%% AASTeX

%% preprint2 produces a double-column, single-spaced document:

%% \documentclass[preprint]{aastex}

%% Sometimes a paper's abstract is too long to fit on the
%% title page in preprint2 mode. When that is the case,
%% use the longabstract style option.

%% \documentclass[preprint2,longabstract]{aastex}

%% If you want to create your own macros, you can do so
%% using \newcommand. Your macros should appear before
%% the \begin{document} command.
%%
%% If you are submitting to a journal that translates manuscripts
%% into SGML, you need to follow certain guidelines when preparing
%% your macros. See the AASTeX v5.x Author Guide
%% for information.

\newcommand{\vdag}{(v)^\dagger}
\newcommand{\myemail}{jack.ireland@nasa.gov}
\newcommand{\PS}{power spectrum}
\newcommand{\mPS}{mean power spectrum}

\newcommand{\PA}{power spectra}
\newcommand{\mPA}{mean power spectra}

\newcommand{\PL}{power-law}
\newcommand{\PLs}{power-laws}

\newcommand{\Fps}{Fourier \PS}
\newcommand{\mFps}{mean Fourier \PS}

\newcommand{\Fpa}{Fourier \PA}
\newcommand{\mFpa}{mean Fourier \PA}

\newcommand{\allaiaeuv}{171\AA, 193\AA, 211\AA, 335\AA, 94\AA\ and 131\AA}

\newcommand{\BF}{ }

%% You can insert a short comment on the title page using the command below.

%\slugcomment{Not to appear in Nonlearned J., 45.}

%% If you wish, you may supply running head information, although
%% this information may be modified by the editorial offices.
%% The left head contains a list of authors,
%% usually a maximum of three (otherwise use et al.).  The right
%% head is a modified title of up to roughly 44 characters.
%% Running heads will not print in the manuscript style.

\shorttitle{Spatial variation of coronal Fourier power spectra}
\shortauthors{Ireland et al.}

%% This is the end of the preamble.  Indicate the beginning of the
%% paper itself with \begin{document}.

\begin{document}

%% LaTeX will automatically break titles if they run longer than
%% one line. However, you may use \\ to force a line break if
%% you desire.

\title{Spatial variation of coronal Fourier power spectra}

%% Use \author, \affil, and the \and command to format
%% author and affiliation information.
%% Note that \email has replaced the old \authoremail command
%% from AASTeX v4.0. You can use \email to mark an email address
%% anywhere in the paper, not just in the front matter.
%% As in the title, use \\ to force line breaks.

\author{J. Ireland}
\affil{ADNET Systems, Inc.,NASA Goddard Space Flight Center, MC. 671.1, Greenbelt, MD 20771, USA.}

\author{R. T. J. McAteer}
\affil{Department of Astronomy, New Mexico State University, Las
  Cruces, NM.}


%% Mark off your abstract in the ``abstract'' environment. In the manuscript
%% style, abstract will output a Received/Accepted line after the
%% title and affiliation information. No date will appear since the author
%% does not have this information. The dates will be filled in by the
%% editorial office after submission.

\begin{abstract}
The dynamics of the solar corona are investigated using data from the
six extreme ultraviolet channels (EUV) of the  Atmospheric Imaging
Assembly (AIA) of the Solar Dynamics Observatory (SDO).

It is shown that in almost all locations the \Fps\ are well be described by
a \PL\ at lower frequencies that tails to flat spectrum at higher
frequencies, plus a Gaussian-shaped contribution that varies depending
on the region studied.  The spatial distribution of the power law
index is described as a function of AIA EUV channel.


\end{abstract}

%% Keywords should appear after the \end{abstract} command. The uncommented
%% example has been keyed in ApJ style. See the instructions to authors
%% for the journal to which you are submitting your paper to determine
%% what keyword punctuation is appropriate.

\keywords{Sun: corona, Sun: oscillations, methods: data analysis,
  methods: statistical}

%% From the front matter, we move on to the body of the paper.
%% In the first two sections, notice the use of the natbib \citep
%% and \citet commands to identify citations.  The citations are
%% tied to the reference list via symbolic KEYs. The KEY corresponds
%% to the KEY in the \bibitem in the reference list below. We have
%% chosen the first three characters of the first author's name plus
%% the last two numeral of the year of publication as our KEY for
%% each reference.


%% Authors who wish to have the most important objects in their paper
%% linked in the electronic edition to a data center may do so by tagging
%% their objects with \objectname{} or \object{}.  Each macro takes the
%% object name as its required argument. The optional, square-bracket 
%% argument should be used in cases where the data center identification
%% differs from what is to be printed in the paper.  The text appearing 
%% in curly braces is what will appear in print in the published paper. 
%% If the object name is recognized by the data centers, it will be linked
%% in the electronic edition to the object data available at the data centers  
%%
%% Note that for sources with brackets in their names, e.g. [WEG2004] 14h-090,
%% the brackets must be escaped with backslashes when used in the first
%% square-bracket argument, for instance, \object[\[WEG2004\] 14h-090]{90}).
%%  Otherwise, LaTeX will issue an error. 

\section{Introduction}\label{sec:int}
Coronal seismology is the study of oscillatory phenomena in the solar
corona.  First suggested by \cite{1970PASJ...22..341U}, the practice
of coronal seismology attempts to locate oscillatory phenomena in the
solar corona, identify the wave modes present, and then use
theoretical descriptions of those wave modes to infer the physical
conditions of the solar corona.  The practical application of coronal
seismological ideas began in earnest with observations of wave modes
{\BF described in \cite{1998ApJ...501L.217D} and
  \cite{1999SoPh..186..207B} from data captured by Extreme Ultraviolet
  Telescope (EIT, \citealp{1995SoPh..162..291D}) on board the Solar and Heliospheric
  Observatory (SOHO, \citealp{1995SoPh..162....1D}), and also by
  \citep{1999ApJ...520..880A} using data from the Transition Region
  and Coronal Explorer (TRACE, \citealp{1999SoPh..187..229H}).  Since these initial works,
  many papers have been published describing oscillatory phenomena in
  the corona and their interpretation in terms of theoretically known
  wave modes.  The review articles \cite{lrsp-2005-3} and
  \cite{2012RSPTA.370.3193D} refer to many more articles on both
  observational and theoretical coronal seismology}.

Oscillatory phenomena in the solar corona are relatively rare, both
spatially and temporally.  Two types of oscillatory motions have
received a lot of attention.  The first is known as a transverse
oscillation, since the loop motion is approximately transverse to the
extent of the loop \citep{1999Sci...285..862N}.  These motions appear
to be triggered by nearby flaring events, and are identified with the
fundamental harmonic of the kink mode of the flux tube that comprises
the coronal loop.  The second type is known as a longitudinal
oscillation, or propagating disturbance (PD).  These waves propagate
along flux tubes, and have been identified as slow mode
magnetohydrodynamic (MHD) magneto-acoustic waves (although alternative
interpretations do exist that do not invoke oscillatory phenomena, for
example, \citealp*{0004-637X-722-2-1013} and
\citealp*{2041-8205-727-2-L37}).  

\cite{ireland2015} analyzed power spectra summed over physically
different extended regions.  It was noted that some regions, when
observed in two different AIA channels result in different power-law
power indices, while other regions have approximately similar
power-law spectral indices.  For example, loop footpoint locations had
very similar spectral index of approximately 2.28, whereas AIA 171\AA\
and 193\AA\ had spectral indices of 1.76 and 2.05 respectively.  It is
therefore clear that different regions exhibit different evolutionary
behavior.  This paper extends that work by fitting models of power
spectra arising from time-series in all six EUV channels (\allaiaeuv)
at all available pixels in the field of view.

Section \ref{sec:obs} describes the observations used, while Section
\ref{sec:anal} describes the method of analysis employed and presents
the results.  Section \ref{sec:discuss} discusses the detected
oscillatory content, and the implications of the observed power-law
power spectra.  Section \ref{sec:conc} summarizes the discussion and
outlines future work.


\section{Observations}\label{sec:obs}
The region studied is the same as that used in the previous
\cite{ireland2015} study.  The selected area covers a single
$\beta\gamma$ active region, NOAA AR 11575.  As was noted in
\cite{ireland2015} , this area is quiescent in that there were no
X-ray flares or other large-scale disturbances (for example, filament
eruptions or rearrangments of long bright coronal loops) over the
spatial extent of the cutout within the time range.
\cite{ireland2015} note the power spectra

{\BF AIA} observes images of the Sun in multiple channels
simultaneously and continuously, at high cadence (12 seconds) and at
high spatial resolution (0.6 arcseconds per pixel).  {\BF AIA} data
was acquired using the Lockheed Martin Solar Astrophysics cutout
service at {\it http://lmsal.com/get\_aia\_data/}.  Six hours of image
data in the time range 2012-09-23 00:00 - 06:00 UT, at the full 12
second cadence, in EUV wavebands \allaiaeuv\ are used.  
\begin{figure}
\centerline{
\epsscale{0.8}\plottwo{paper2_six_euv_disk_1.5_94.exact_map.six_euv.eps}{paper2_six_euv_disk_1.5_131.exact_map.six_euv.eps}
}
\centerline{
\epsscale{0.8}\plottwo{paper2_six_euv_disk_1.5_171.exact_map.six_euv.eps}{paper2_six_euv_disk_1.5_193.exact_map.six_euv.eps}
}
\centerline{
\epsscale{0.8}\plottwo{paper2_six_euv_disk_1.5_211.exact_map.six_euv.eps}{paper2_six_euv_disk_1.5_335.exact_map.six_euv.eps}
}
\caption{Example images from each of the AIA channels studied.}
\label{fig:allimages}
\end{figure}

\section{Analysis}\label{sec:anal}
The data in each AIA channel are co-aligned to remove solar rotation
using the process as described in Appendix \ref{sec:app:data}.  The
result are a set of datacubes that show the same portion of the Sun
for the duration of the data.  The time-series of emission $I(t)$ from
a single pixel is apodized using the Hanning window, reducing ringing
effects in the \PS.  The full 12 second cadence is used, yielding time
series of 1800 samples.  The full spatial resolution is also retained
in order to obtain fine detail on the variation of frequency content
as a function of location.  This yields $333\times566=188478$ spectra
per AIA channel for analysis.

Figure \ref{fig:compare171193} show example power spectra, plotted
using a log-log scale, from a single location inside each of the
regions shown in Figure \ref{fig:exampleFpa}.  The shape of the power
spectra indicate that the Fourier power approximately follows a
\PL\ at lower frequencies, and flattens out at higher frequencies, as
expected from \citet{ireland2015}. 

Following \citet{ireland2015}, two models for the \Fps\ are
considered. Model $M_{1}$ is
\begin{equation}
\label{eqn:pwrlaw}
P_{1}(\nu) = A\nu^{-n} + C,
\end{equation}
where $\nu$ is the frequency, $A>0$ is a proportionality constant and
$n>0$ is the \PL\ index.  The quantity $C>0$ models the high-frequency
/ low power end of the spectrum where the detection properties of the
detector apparatus are assumed to dominate the observations. The
second model $M_{2}$ has two contributions to the overall \Fps; a
background power-law of the form of $P_{1}$, and a contribution that
is more localized, $G(\nu)$.  The model is given by
\begin{equation}
\label{eqn:pwrlawbump}
P_{2}(\nu) = P_{1}(\nu) + G(\nu)
\end{equation}
where the localized contribution is described by
\begin{equation}
\label{eqn:bump}
G(\nu) = \alpha\exp\left[-\frac{(\ln(\nu)-\beta)^{2}}{2\delta^{2}}\right].
\end{equation}

Each Fourier power spectrum is fit using both models $M_{1}$ and
$M_{2}$.  A two-stage filtering process is used to find which model
best describes the Fourier power spectrum of time-series of emission
at each pixel.  The first stage determines if a model fit to the data
is 'good' or not.  A model fit is labeled a 'good fit' if the
following criteria are met: (a) the fitting routine does not report an
error, (b) the $\chi^{2}$-analog lies within the 95\% confidence
interval (see Appendix \ref{sec:app:chi}) and (c) all the parameters
found by the fitting process lie inside acceptable limits (see
Appendix \ref{sec:app:parlimits}).  This qualifies the fit of the
model to the data as worthy of further consideration.

The second stage determines which model is preferred.  Firstly, we
restrict attention to \Fpa\ that have good fits (as defined above) for
all the models considered.  Then, we use the Bayesian Information
Criterion (BIC) determine which model is substabtially preferred to
another.  The model $M_{k}$ is preferred if
\begin{equation}
\label{eqn:bic}
BIC_{k} - BIC_{j} \le -6, \forall j \neq k.
\end{equation}
This selection criterion picks the model that is substantially better
descriptions of the data compared to the other models used.  If no
model passes the selection criterion, then the \Fps\ arising from the
time-series at this pixel fails this criterion and is not retained in
the results below.  Only \Fpa\ that pass this two stage filter process
are retained for further analysis.


\subsection{Results}\label{ssec:results}


\begin{figure}
\label{fig:spatialdistribution:preference}
\centerline{
\epsscale{0.8}\plottwo{paper2_six_euv_disk_1.5_94.exact_map.six_euv.eps}{paper2_six_euv_disk_1.5_131.exact_map.six_euv.eps}
}
\centerline{
\epsscale{0.8}\plottwo{paper2_six_euv_disk_1.5_171.exact_map.six_euv.eps}{paper2_six_euv_disk_1.5_193.exact_map.six_euv.eps}
}
\centerline{
\epsscale{0.8}\plottwo{paper2_six_euv_disk_1.5_211.exact_map.six_euv.eps}{paper2_six_euv_disk_1.5_335.exact_map.six_euv.eps}
}
\caption{Spatial distribution of model preference.}
\end{figure}



\begin{figure}
\label{fig:spatialdistribution:powerlawindex}
\centerline{
\epsscale{0.8}\plottwo{paper2_six_euv_disk_1.5_94.exact_map.six_euv.eps}{paper2_six_euv_disk_1.5_131.exact_map.six_euv.eps}
}
\centerline{
\epsscale{0.8}\plottwo{paper2_six_euv_disk_1.5_171.exact_map.six_euv.eps}{paper2_six_euv_disk_1.5_193.exact_map.six_euv.eps}
}
\centerline{
\epsscale{0.8}\plottwo{paper2_six_euv_disk_1.5_211.exact_map.six_euv.eps}{paper2_six_euv_disk_1.5_335.exact_map.six_euv.eps}
}
\caption{Spatial distribution of the power law indices.  Power law
  indices are taken from either model $M_{1}$ or $M_{2}$, whichever
  model is preferred at each pixel (see Section \ref{sec:anal})}
\end{figure}

\begin{figure}
\label{fig:spatialdistribution:gaussianposition}

\caption{Spatial distribution of the Gaussian position,}
\end{figure}


\begin{figure}
\label{fig:spatialdistribution:gaussianwidth}

\caption{Spatial distribution of the Gaussian width.}
\end{figure}


\begin{figure}
\label{fig:spatialdistribution:gaussianamplitude}

\caption{Spatial distribution of the ratio $\max[P_{1}(\nu)/G(\nu)]$.}
\end{figure}

\begin{figure}
\label{fig:spatialdistribution:gaussianamplitude}

\caption{Spatial distribution of $\arg\max[P_{1}(\nu)/G(\nu)]$.}
\end{figure}



\begin{figure}
\label{fig:powerlawindicescc}

\caption{Two dimensional histograms of the power law indices in every
  pair of AIA EUV channels.  The red line indicates the line where the
  power law indices are equal. Table \ref{tab:powerlawindicescc} shows
the corresponding cross correlation coefficients and the likelihood
that such a cross-correlation coefficient occurred randomly.}
\end{figure}




\section{Discussion}\label{sec:discuss}
The presence of a \PL\ \PS\ has implications for the detection of
narrow frequency band oscillations against such a background emission,
for both non-automated and automated methods.  These are discussed in
Sections \ref{ssec:corseis} and \ref{sec:oscdetect}.  Further, the
\PL\ Fourier spectrum in coronal emission poses questions about how
that emission is formed.  Section \ref{ssec:excess} discusses possible
reasons for the localized contribution component $G(\nu)$.  A
hypothesis regarding the formation of this \PL\ spectrum is
presented in Section \ref{ssec:nplps}.

PSF removal

De noising algorithms


%% If you wish to include an acknowledgments section in your paper,
%% separate it off from the body of the text using the \acknowledgments
%% command.

%% Included in this acknowledgments section are examples of the
%% AASTeX hypertext markup commands. Use \url without the optional [HREF]
%% argument when you want to print the url directly in the text. Otherwise,
%% use either \url or \anchor, with the HREF as the first argument and the
%% text to be printed in the second.

\acknowledgments
We are grateful to the developers of SSWIDL
\citep{1998SoPh..182..497F}, IPython \citep{ipython}, SunPy
\citep{mumford-proc-scipy-2013}, matplotlib
\citep{Hunter:2007} and the Scientific Python stack for providing data
preparation, manipulation, analysis and display packages.  This work
was supported by NASA award NNX13AE03G S01 funded through NASA ROSES
NNH12ZDA001N-SHP, and by a NSF Career grant 1255024 (JMA).


%% To help institutions obtain information on the effectiveness of their
%% telescopes, the AAS Journals has created a group of keywords for telescope
%% facilities. A common set of keywords will make these types of searches
%% significantly easier and more accurate. In addition, they will also be
%% useful in linking papers together which utilize the same telescopes
%% within the framework of the National Virtual Observatory.
%% See the AASTeX Web site at http://aastex.aas.org/
%% for information on obtaining the facility keywords.

%% After the acknowledgments section, use the following syntax and the
%% \facility{} macro to list the keywords of facilities used in the research
%% for the paper.  Each keyword will be checked against the master list during
%% copy editing.  Individual instruments or configurations can be provided 
%% in parentheses, after the keyword, but they will not be verified.

{\it Facilities:} \facility{SDO (AIA)}.

%% Appendix material should be preceded with a single \appendix command.
%% There should be a \section command for each appendix. Mark appendix
%% subsections with the same markup you use in the main body of the paper.

%% Each Appendix (indicated with \section) will be lettered A, B, C, etc.
%% The equation counter will reset when it encounters the \appendix
%% command and will number appendix equations (A1), (A2), etc.

\appendix\section{Data preparation and model
  fitting}\label{sec:app:data}

The data was prepared for analysis using the SolarSoft / IDL routines
READ\_SDO and AIA\_PREP.  These procedures convert the downloaded
level 1.0 FITS cutout files to level 1.5 FITS files.  This involves
translating, scaling and rotating the images so they have the same
sun-center, image scale with solar north in the same direction.  All
further processing from this point was performed using SunPy 0.7
(http://www.sunpy.org) running under Python 3.5.  

Images were de-rotated to compensate for solar rotation, based on the
center of the field of view of each image.  After de-rotation, each
image layer was coaligned with a reference layer, defined to be half
way through the dataset, in this case at approximately 03:00 UT.  This
process is required since the solar de-rotation function used does not
fully compensate for all the solar rotation. Image coalignment was
implemented using a template matching technique \citep{lewis1995fast}
implemented by the {\it scikit-image} \citep{Vanderwalt2014} routine
{\it match\_template}.  A rectangular template sub-image with sides
one half the size (in pixels) of the reference layer is defined.  The
coalignment algorithm finds the location of the best match of this
template with each layer.  The image layer is then shifted (using
sub-pixel shifts) according to the location of the best match of the
template with the image.  This process removes residual bulk motions
of the images.

In all AIA channels except 131\AA\ the measured intensity of the data
was used to calculate the shifts required for co-alignment.  For
131\AA\ the square root of the measured intensity was used to
calculate the shifts.  This was necessary due to the appearance of
small, localized intense brightening at time ??? - ??? that caused the
template matching technique to fail in 131\AA\ when using the measured
intensity.

%% The image shifts caused by the co-alignment routine are shown in
%% Figure \ref{fig:coalignment}.  Note that there is rapid and sudden
%% change in the shifts around ???-???, corresponding to ??? frames in
%% each channel's data cube.  Examination by eye of the movies of the
%% data cubes before this co-alignment step shows that there is a
%% noticeable shift in all features in the field of view as suggested by
%% the image shift plot Figure \ref{fig:coalignment}.  Examination by eye
%% of the movies of the data cubes after the co-alignment shifts have
%% been applied shows no noticeable shift, leading to the conclusion that
%% the template matching routine has corrected for the shift adequately.
%% As a final check, the analysis detailed in Section \ref{???}  was
%% applied to the co-aligned datacubes with the final ??? frames removed.
%% Excluding these frames did not significantly affect the final results
%% or conclusions.

\subsection{Fitting the data}\label{sec:app:fit}
Section \ref{sec:anal} describes two models of the Fourier power
spectrum that are fit to the observed Fourier power spectrum
\Fpa\ models are fit by first generating an estimate of the model
parameters.  This estimate is used as an initial guess to a fitting
routine.

The initial guess for model \ref{eqn:pwrlaw} is generated as follows.
A power law index $n$ is estimated using the observed Fourier power
spectra at the fifty lowest non-zero frequencies using the estimation
approach detailed in Section \ref{sec:app:est}.  For the six hours of
data studied, this covers the frequency range $0.046\rightarrow2.36
mHz$.  The fifty lowest non-zero frequencies are used as this is
typically the frequency range where the observed \Fps\ is dominated by
a power-law like behavior.  An estimate for the power law amplitude
$A_{p}$ is found be averaging the observed power at the first five
non-zero frequencies, covering the frequency range
$0.046\rightarrow0.28 mHz$.  An estimate for the background constant
value $C$ is defined by taking the exponential of the average of the
logarithm of the observed power over the last fifty Fourier
frequencies, covering the frequency range $39.35\rightarrow41.6 mHz$.
Averaging over the logarithm of the power reduces the chance that the
estimate of $C$ will be strongly biased by the largest power in the
frequency range.

The initial guess for model \ref{eqn:pwrlawbump} is generated as
follows.  Estimates for $n$, $A_{p}$ and $C$ are generated as
described in the previous paragraph.  An estimated power spectrum
$P_{est}$ is then calculated using these estimates and model
\ref{eqn:pwrlaw}.  If there at least ten positive differences between
the observed Fourier power and estimated power spectrum $P_{est}$ in
the frequency range $0.97 \rightarrow 9.31 mHz$, then these positive
differences are used to estimate the parameters of the Gaussian,
equation \ref{eqn:bump}.  The parameters $A_{G}$ and $\beta$ are
estimated as the maximum value of the difference and the logarithm of
the Fourier frequency at which this maximum value occurs,
respectively.  The width $\sigma$ is estimated by calculating the
standard deviation of the positive differences with respect to the
logarithm of the frequency.  If there are less than ten positive
differences in the required frequency range, it is assumed that there
is a very small Gaussian contribution to the observed \Fps.  In this
case, $\ln A_{G}=-100$, $\beta$ is fixed to be half way between the
logarithm lower and upper estimated Gaussian frequency range
$0.97 \rightarrow 9.31 mHz$, and $\sigma = 0.1$.

Model fitting was initialized using the initial estimates described
above.  Models were fit by varying the value of model parameters in
order to find the maximum of the log of likelihood of the data given
the model.  If the time-series data are subject to a Gaussian noise,
then the Fourier power components are exponentially distributed.  Let
us assume that $I_{j}$ is the observed Fourier power at the frequency
indexed by $j$, and $S_{j}$ is the model Fourier power at that same
frequency.  The probability of an observed power $I_{j}$ given a model
power $S_{j}$ is given by
\begin{equation}
p_{j}\left(I_{j}|S_{j}\right) =
\frac{1}{S_{j}}\exp\left[-\frac{I_{j}}{S_{j}}\right]
\label{eqn:freqprob}
\end{equation}
If there are $N$ non-zero Fourier frequencies then the likelihood over
all frequencies is
\begin{equation}
L=\prod_{j=1}^{N}p\left(I_{j}|S_{j}\right)
\label{eqn:freqprob:all}
\end{equation}
The Nelder-Mead method was used to vary model parameters in search of
the maximum of the log likelihood.  The full code used (and the
history of its development) to generate the results in this paper is
available as a git repository at
https://github.com/wafels/rednoise/tree/perpixel.

\subsection{Estimating the power-law index in a power law power spectrum}\label{sec:app:est}
Equation \ref{eqn:freqprob:all} describes the likelihood of the
observed Fourier power given a model for that power.  We assume that
the observed power law spectrum can be modeled by a simple power law
over all frequencies
\begin{equation}
S_{j} = Af_{j}^{-n}
\end{equation}
where $A$ is a constant, $f_{j}$ is the $j$'th Fourier frequency, and
$n$ is the power law index we wish to estimate.  An estimate can be
derived using Bayes' theorem. Using the likelihood function derived
above (\ref{eqn:freqprob:all}), and assuming the prior $p(A)\propto
A^{m}$ for some value of $m$, then 
\begin{equation}
p(n) \propto \int_{0}^{\infty} p(A) L dA.
\label{eqn:bayes}
\end{equation}
On integration,
\begin{equation}
\ln p(n) \propto n\sum_{j=1}^{N}\ln f_{j} - (N-m-1)\ln\left[\sum_{j=1}^{N}I_{j}f_{j}
\right]
\label{eqn:pnlog}
\end{equation}
For the purposes of this investigation, $m=0$ and an estimate to $n$
was found by finding calculating the value of $\ln p(n)$ in the range
$n=0\rightarrow4$ with steps of 0.01 and finding the value of $n$ that
maximized $\ln p(n)$.

\subsection{Calculating the 95\% confidence interval using the $\chi^{2}$
analog}\label{sec:app:chi} The fitting process finds model parameter
values that approximate the shape of the \Fps.  Noise in \Fps\ is
exponentially distributed, and so the reduced-$\chi^{2}$ measure
cannot be used to assess how well the model fits the
data. \citet{2014ApJ...789..152N} derive an analogous
reduced-$\chi^{2}$ for exponentially distributed data such as \Fpa.
Expected values of the reduced-$\chi^{2}$ analogue are calculated by
inverting Equation 18 of \citet{2014ApJ...789..152N} to make the
reduced-$\chi^{2}$ analogue the subject of the equation.  The 95\%
confidence interval is found by calculating the values of the
reduced-$\chi^{2}$ analogue that exclude the top and bottom 2.5\% of
the probability density function (Equation 18,
\citet{2014ApJ...789..152N}).

\subsection{Acceptable parameter limits}\label{sec:app:limit}
Even after the filtering process described in Section \ref{sec:anal},
the fitting algorithm still finds values which are probably outliers.
In order to present the vast majority of the results, limits are set
to the model parameters.  If any of the model parameters violate these
limits, the model fit to the data is classed as a bad fit and is not
considered further.  The table below lists the range of parameter
values, and gives the percentage of the model fits which violate these
limits as a function of AIA channel.  The fraction is calculated as
the number of model fits that violate these limits divided by the
total number of model fits.

\begin{deluxetable}{cccccccc}
\tabletypesize{\scriptsize} \tablecolumns{8} \tablewidth{0pt}
\tablecaption{Acceptable parameter ranges and the fraction of model
  fits that exceed the parameter ranges.  Section \ref{sec:anal} describe the meaning of each parameter value\label{tab:acceptable}}
\tablehead{
  \multicolumn{1}{c}{Parameter} & \multicolumn{1}{c}{Range} & \multicolumn{6}{c}{AIA channel}\\
  \colhead{ } & \colhead{ } & \colhead{94\AA} & \colhead{131\AA} & \colhead{171\AA} & \colhead{193\AA}  & \colhead{211\AA} & \colhead{335\AA} \\ }
\startdata 
$A$ & ???$\rightarrow$??? & 94 & 131 & 171 & 193 & 211 & 335 \\
$n$ & ???$\rightarrow$??? & 94 & 131 & 171 & 193 & 211 & 335 \\
$C$ & ???$\rightarrow$??? & 94 & 131 & 171 & 193 & 211 & 335 \\
$\alpha$ & ???$\rightarrow$??? & 94 & 131 & 171 & 193 & 211 & 335 \\
$\log_{10}(\beta)$ & ???$\rightarrow$??? & 94 & 131 & 171 & 193 & 211 & 335 \\
$\log_{10}(\delta)$ & ???$\rightarrow$??? & 94 & 131 & 171 & 193 & 211 & 335 \\
\enddata
%% Text for table notes should follow after the \enddata but before
%% the \end{deluxetable}. Make sure there is at least one \tablenotemark
%% in the table for each \tablenotetext.

\end{deluxetable}



%% The reference list follows the main body and any appendices.
%% Use LaTeX's thebibliography environment to mark up your reference list.
%% Note \begin{thebibliography} is followed by an empty set of
%% curly braces.  If you forget this, LaTeX will generate the error
%% "Perhaps a missing \item?".
%%
%% thebibliography produces citations in the text using \bibitem-\cite
%% cross-referencing. Each reference is preceded by a
%% \bibitem command that defines in curly braces the KEY that corresponds
%% to the KEY in the \cite commands (see the first section above).
%% Make sure that you provide a unique KEY for every \bibitem or else the
%% paper will not LaTeX. The square brackets should contain
%% the citation text that LaTeX will insert in
%% place of the \cite commands.

%% We have used macros to produce journal name abbreviations.
%% AASTeX provides a number of these for the more frequently-cited journals.
%% See the Author Guide for a list of them.

%% Note that the style of the \bibitem labels (in []) is slightly
%% different from previous examples.  The natbib system solves a host
%% of citation expression problems, but it is necessary to clearly
%% delimit the year from the author name used in the citation.
%% See the natbib documentation for more details and options.

\bibliography{references}
\end{document}
